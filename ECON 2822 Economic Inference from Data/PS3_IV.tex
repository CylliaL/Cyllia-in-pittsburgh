% Options for packages loaded elsewhere
\PassOptionsToPackage{unicode}{hyperref}
\PassOptionsToPackage{hyphens}{url}
%
\documentclass[
]{article}
\usepackage{amsmath,amssymb}
\usepackage{iftex}
\ifPDFTeX
  \usepackage[T1]{fontenc}
  \usepackage[utf8]{inputenc}
  \usepackage{textcomp} % provide euro and other symbols
\else % if luatex or xetex
  \usepackage{unicode-math} % this also loads fontspec
  \defaultfontfeatures{Scale=MatchLowercase}
  \defaultfontfeatures[\rmfamily]{Ligatures=TeX,Scale=1}
\fi
\usepackage{lmodern}
\ifPDFTeX\else
  % xetex/luatex font selection
\fi
% Use upquote if available, for straight quotes in verbatim environments
\IfFileExists{upquote.sty}{\usepackage{upquote}}{}
\IfFileExists{microtype.sty}{% use microtype if available
  \usepackage[]{microtype}
  \UseMicrotypeSet[protrusion]{basicmath} % disable protrusion for tt fonts
}{}
\makeatletter
\@ifundefined{KOMAClassName}{% if non-KOMA class
  \IfFileExists{parskip.sty}{%
    \usepackage{parskip}
  }{% else
    \setlength{\parindent}{0pt}
    \setlength{\parskip}{6pt plus 2pt minus 1pt}}
}{% if KOMA class
  \KOMAoptions{parskip=half}}
\makeatother
\usepackage{xcolor}
\usepackage[margin=1in]{geometry}
\usepackage{color}
\usepackage{fancyvrb}
\newcommand{\VerbBar}{|}
\newcommand{\VERB}{\Verb[commandchars=\\\{\}]}
\DefineVerbatimEnvironment{Highlighting}{Verbatim}{commandchars=\\\{\}}
% Add ',fontsize=\small' for more characters per line
\usepackage{framed}
\definecolor{shadecolor}{RGB}{248,248,248}
\newenvironment{Shaded}{\begin{snugshade}}{\end{snugshade}}
\newcommand{\AlertTok}[1]{\textcolor[rgb]{0.94,0.16,0.16}{#1}}
\newcommand{\AnnotationTok}[1]{\textcolor[rgb]{0.56,0.35,0.01}{\textbf{\textit{#1}}}}
\newcommand{\AttributeTok}[1]{\textcolor[rgb]{0.13,0.29,0.53}{#1}}
\newcommand{\BaseNTok}[1]{\textcolor[rgb]{0.00,0.00,0.81}{#1}}
\newcommand{\BuiltInTok}[1]{#1}
\newcommand{\CharTok}[1]{\textcolor[rgb]{0.31,0.60,0.02}{#1}}
\newcommand{\CommentTok}[1]{\textcolor[rgb]{0.56,0.35,0.01}{\textit{#1}}}
\newcommand{\CommentVarTok}[1]{\textcolor[rgb]{0.56,0.35,0.01}{\textbf{\textit{#1}}}}
\newcommand{\ConstantTok}[1]{\textcolor[rgb]{0.56,0.35,0.01}{#1}}
\newcommand{\ControlFlowTok}[1]{\textcolor[rgb]{0.13,0.29,0.53}{\textbf{#1}}}
\newcommand{\DataTypeTok}[1]{\textcolor[rgb]{0.13,0.29,0.53}{#1}}
\newcommand{\DecValTok}[1]{\textcolor[rgb]{0.00,0.00,0.81}{#1}}
\newcommand{\DocumentationTok}[1]{\textcolor[rgb]{0.56,0.35,0.01}{\textbf{\textit{#1}}}}
\newcommand{\ErrorTok}[1]{\textcolor[rgb]{0.64,0.00,0.00}{\textbf{#1}}}
\newcommand{\ExtensionTok}[1]{#1}
\newcommand{\FloatTok}[1]{\textcolor[rgb]{0.00,0.00,0.81}{#1}}
\newcommand{\FunctionTok}[1]{\textcolor[rgb]{0.13,0.29,0.53}{\textbf{#1}}}
\newcommand{\ImportTok}[1]{#1}
\newcommand{\InformationTok}[1]{\textcolor[rgb]{0.56,0.35,0.01}{\textbf{\textit{#1}}}}
\newcommand{\KeywordTok}[1]{\textcolor[rgb]{0.13,0.29,0.53}{\textbf{#1}}}
\newcommand{\NormalTok}[1]{#1}
\newcommand{\OperatorTok}[1]{\textcolor[rgb]{0.81,0.36,0.00}{\textbf{#1}}}
\newcommand{\OtherTok}[1]{\textcolor[rgb]{0.56,0.35,0.01}{#1}}
\newcommand{\PreprocessorTok}[1]{\textcolor[rgb]{0.56,0.35,0.01}{\textit{#1}}}
\newcommand{\RegionMarkerTok}[1]{#1}
\newcommand{\SpecialCharTok}[1]{\textcolor[rgb]{0.81,0.36,0.00}{\textbf{#1}}}
\newcommand{\SpecialStringTok}[1]{\textcolor[rgb]{0.31,0.60,0.02}{#1}}
\newcommand{\StringTok}[1]{\textcolor[rgb]{0.31,0.60,0.02}{#1}}
\newcommand{\VariableTok}[1]{\textcolor[rgb]{0.00,0.00,0.00}{#1}}
\newcommand{\VerbatimStringTok}[1]{\textcolor[rgb]{0.31,0.60,0.02}{#1}}
\newcommand{\WarningTok}[1]{\textcolor[rgb]{0.56,0.35,0.01}{\textbf{\textit{#1}}}}
\usepackage{longtable,booktabs,array}
\usepackage{calc} % for calculating minipage widths
% Correct order of tables after \paragraph or \subparagraph
\usepackage{etoolbox}
\makeatletter
\patchcmd\longtable{\par}{\if@noskipsec\mbox{}\fi\par}{}{}
\makeatother
% Allow footnotes in longtable head/foot
\IfFileExists{footnotehyper.sty}{\usepackage{footnotehyper}}{\usepackage{footnote}}
\makesavenoteenv{longtable}
\usepackage{graphicx}
\makeatletter
\def\maxwidth{\ifdim\Gin@nat@width>\linewidth\linewidth\else\Gin@nat@width\fi}
\def\maxheight{\ifdim\Gin@nat@height>\textheight\textheight\else\Gin@nat@height\fi}
\makeatother
% Scale images if necessary, so that they will not overflow the page
% margins by default, and it is still possible to overwrite the defaults
% using explicit options in \includegraphics[width, height, ...]{}
\setkeys{Gin}{width=\maxwidth,height=\maxheight,keepaspectratio}
% Set default figure placement to htbp
\makeatletter
\def\fps@figure{htbp}
\makeatother
\setlength{\emergencystretch}{3em} % prevent overfull lines
\providecommand{\tightlist}{%
  \setlength{\itemsep}{0pt}\setlength{\parskip}{0pt}}
\setcounter{secnumdepth}{5}
\ifLuaTeX
  \usepackage{selnolig}  % disable illegal ligatures
\fi
\IfFileExists{bookmark.sty}{\usepackage{bookmark}}{\usepackage{hyperref}}
\IfFileExists{xurl.sty}{\usepackage{xurl}}{} % add URL line breaks if available
\urlstyle{same}
\hypersetup{
  pdftitle={Problem Set 3: Instrumental Variables Key},
  pdfauthor={Claire Duquennois},
  hidelinks,
  pdfcreator={LaTeX via pandoc}}

\title{Problem Set 3: Instrumental Variables Key}
\author{Claire Duquennois}
\date{}

\begin{document}
\maketitle

\textbf{\emph{Name: }} Xiang Li

\hypertarget{empirical-analysis-using-data-from-ananat-2011-aejae}{%
\section{1. Empirical Analysis using Data from Ananat (2011,
AEJ:AE)}\label{empirical-analysis-using-data-from-ananat-2011-aejae}}

This exercise uses data from Elizabeth Ananat's paper, ``The Wrong
Side(s) of the Tracks: The Causal Effects of Racial Segregation on Urban
Poverty and Inequality,'' published in the \emph{American Economic
Journal: Applied Economics} in 2011. This paper studies how segregation
has affected population characteristics and income disparity in US
cities using the layout of railroad tracks as an instrumental variable.

\hypertarget{finding-the-data}{%
\section{2. Finding the data}\label{finding-the-data}}

The data can be found by following the link on the AEJ: Applied
Economics' website which will take you to the ICPSR's data repository.
You will need to sign in to get access to the data files. Once logged
in, you will find the set of files that are typically included in a
replication file. These include several datasets, several .do files
(which is a STATA command file). For this assignment we will be using
the\texttt{aej\_maindata.dta} file.

\clearpage

\hypertarget{set-up-and-opening-the-data}{%
\section{3. Set up and opening the
data}\label{set-up-and-opening-the-data}}

\hypertarget{question-load-any-packages-you-will-need-and-the-data-contained-in-the-aej_maindata.dta-file.-how-many-observations-are-contained-in-the-data.-what-is-the-level-of-an-observation}{%
\subsection{\texorpdfstring{3.1 Question: Load any packages you will
need and the data contained in the \texttt{aej\_maindata.dta} file. How
many observations are contained in the data. What is the level of an
observation?}{3.1 Question: Load any packages you will need and the data contained in the aej\_maindata.dta file. How many observations are contained in the data. What is the level of an observation?}}\label{question-load-any-packages-you-will-need-and-the-data-contained-in-the-aej_maindata.dta-file.-how-many-observations-are-contained-in-the-data.-what-is-the-level-of-an-observation}}

\textbf{Code and Answer:}

\begin{Shaded}
\begin{Highlighting}[]
\FunctionTok{library}\NormalTok{(}\StringTok{"haven"}\NormalTok{)}
\FunctionTok{library}\NormalTok{(}\StringTok{"dplyr"}\NormalTok{)}
\end{Highlighting}
\end{Shaded}

\begin{verbatim}
## 
## 载入程辑包:'dplyr'
\end{verbatim}

\begin{verbatim}
## The following objects are masked from 'package:stats':
## 
##     filter, lag
\end{verbatim}

\begin{verbatim}
## The following objects are masked from 'package:base':
## 
##     intersect, setdiff, setequal, union
\end{verbatim}

\begin{Shaded}
\begin{Highlighting}[]
\FunctionTok{library}\NormalTok{(}\StringTok{"stargazer"}\NormalTok{)}
\end{Highlighting}
\end{Shaded}

\begin{verbatim}
## 
## Please cite as:
\end{verbatim}

\begin{verbatim}
##  Hlavac, Marek (2022). stargazer: Well-Formatted Regression and Summary Statistics Tables.
\end{verbatim}

\begin{verbatim}
##  R package version 5.2.3. https://CRAN.R-project.org/package=stargazer
\end{verbatim}

\begin{Shaded}
\begin{Highlighting}[]
\FunctionTok{library}\NormalTok{(}\StringTok{"lfe"}\NormalTok{)}
\end{Highlighting}
\end{Shaded}

\begin{verbatim}
## 载入需要的程辑包:Matrix
\end{verbatim}

\begin{verbatim}
## Warning: 程辑包'Matrix'是用R版本4.3.2 来建造的
\end{verbatim}

\begin{Shaded}
\begin{Highlighting}[]
\NormalTok{maindata }\OtherTok{\textless{}{-}} \FunctionTok{read\_dta}\NormalTok{(}\StringTok{"aej\_maindata.dta"}\NormalTok{)}
\FunctionTok{nrow}\NormalTok{(maindata)}
\end{Highlighting}
\end{Shaded}

\begin{verbatim}
## [1] 121
\end{verbatim}

Answer: There are 121 observations are contained in the data, it's a
city level of an observation.

\clearpage

\hypertarget{data-description}{%
\section{4. Data Description}\label{data-description}}

\hypertarget{questionthe-dataset-contains-many-variables-some-of-which-are-not-used-in-this-exercise.-keep-the-following-variables-in-the-final-dataset-hint-use-the-select-function-in-dplyr.}{%
\subsection{\texorpdfstring{4.1 Question:The dataset contains many
variables, some of which are not used in this exercise. Keep the
following variables in the final dataset (Hint: use the \texttt{select}
function in
\texttt{dplyr}).}{4.1 Question:The dataset contains many variables, some of which are not used in this exercise. Keep the following variables in the final dataset (Hint: use the select function in dplyr).}}\label{questionthe-dataset-contains-many-variables-some-of-which-are-not-used-in-this-exercise.-keep-the-following-variables-in-the-final-dataset-hint-use-the-select-function-in-dplyr.}}

\begin{longtable}[]{@{}
  >{\raggedright\arraybackslash}p{(\columnwidth - 2\tabcolsep) * \real{0.1220}}
  >{\raggedright\arraybackslash}p{(\columnwidth - 2\tabcolsep) * \real{0.8780}}@{}}
\toprule\noalign{}
\begin{minipage}[b]{\linewidth}\raggedright
Name
\end{minipage} & \begin{minipage}[b]{\linewidth}\raggedright
Description
\end{minipage} \\
\midrule\noalign{}
\endhead
\bottomrule\noalign{}
\endlastfoot
dism1990 & 1990 dissimilarity index \\
herf & RDI (Railroad division index) \\
lenper & Track length per square km \\
povrate\_w & White poverty rate 1990 \\
povrate\_b & Black poverty rate 1990 \\
area1910 & Physical area in 1910 (1000 sq. miles) \\
count1910 & Population in 1910 (1000s) \\
ethseg10 & Ethnic Dissimilariy index in 1910 \\
ethiso10 & Ethnic isolation index in 1910 \\
black1910 & Percent Black in 1910 \\
passpc & Street cars per capita 1915 \\
black1920 & Percent Black 1920 \\
lfp1920 & Labor Force Participation 1920 \\
incseg & Income segregation 1990 \\
pctbk1990 & Percent Black 1990 \\
manshr & Share employed in manufacturing 1990 \\
pop1990 & Population in 1990 \\
\end{longtable}

\textbf{You can find the detailed description of each variable in the
original paper. }

\textbf{Code:}

\begin{Shaded}
\begin{Highlighting}[]
\NormalTok{dataselect }\OtherTok{\textless{}{-}} \FunctionTok{select}\NormalTok{(maindata, dism1990, herf, lenper, povrate\_w, povrate\_b, area1910, count1910, ethseg10, ethiso10, black1910, passpc, black1920, lfp1920, incseg, pctbk1990, manshr, pop1990)}
\end{Highlighting}
\end{Shaded}

\clearpage

\hypertarget{summary-statistics}{%
\section{5. Summary Statistics:}\label{summary-statistics}}

\hypertarget{question-report-summary-statistics-of-the-following-variables-in-the-datasetdism1990-herf-lenper-povrate_w-povrate_b.-present-these-summary-statistics-in-a-formatted-table-you-can-use-stargazer-or-other-packages.}{%
\subsection{\texorpdfstring{5.1 Question: Report summary statistics of
the following variables in the dataset:``dism1990'', ``herf'',
``lenper'', ``povrate\_w'', ``povrate\_b''. Present these summary
statistics in a formatted table, you can use \texttt{stargazer} or other
packages.}{5.1 Question: Report summary statistics of the following variables in the dataset:``dism1990'', ``herf'', ``lenper'', ``povrate\_w'', ``povrate\_b''. Present these summary statistics in a formatted table, you can use stargazer or other packages.}}\label{question-report-summary-statistics-of-the-following-variables-in-the-datasetdism1990-herf-lenper-povrate_w-povrate_b.-present-these-summary-statistics-in-a-formatted-table-you-can-use-stargazer-or-other-packages.}}

\textbf{Code:}

\begin{Shaded}
\begin{Highlighting}[]
\NormalTok{data }\OtherTok{\textless{}{-}} \FunctionTok{as.data.frame}\NormalTok{(dataselect)}
\NormalTok{selected\_vars }\OtherTok{\textless{}{-}} \FunctionTok{c}\NormalTok{(}\StringTok{"dism1990"}\NormalTok{, }\StringTok{"herf"}\NormalTok{, }\StringTok{"lenper"}\NormalTok{, }\StringTok{"povrate\_w"}\NormalTok{, }\StringTok{"povrate\_b"}\NormalTok{)}
\FunctionTok{stargazer}\NormalTok{(data[, selected\_vars], }\AttributeTok{type =} \StringTok{"text"}\NormalTok{, }\AttributeTok{digits=}\DecValTok{3}\NormalTok{)}
\end{Highlighting}
\end{Shaded}

\begin{verbatim}
## 
## =========================================
## Statistic  N  Mean  St. Dev.  Min    Max 
## -----------------------------------------
## dism1990  121 0.569  0.135   0.329  0.873
## herf      121 0.723  0.141   0.238  0.987
## lenper    121 0.001  0.001   0.0002 0.013
## povrate_w 121 0.095  0.035   0.035  0.216
## povrate_b 121 0.264  0.080   0.093  0.504
## -----------------------------------------
\end{verbatim}

\clearpage

\hypertarget{reduced-form}{%
\section{6. Reduced Form:}\label{reduced-form}}

\hypertarget{question-we-are-interested-in-understanding-how-segregation-affects-population-characteristics-and-income-disparity-in-us-cities.-we-will-focus-on-two-outcome-variables-the-poverty-rate-for-blacks-and-whites.-regress-these-two-outcome-variables-on-segregation-in-1990-our-explanatory-variable-and-interpret-your-results.-report-robust-standard-errors.-make-sure-you-specify-the-units-of-measurement-in-your-interpretation.}{%
\subsection{6.1 Question: We are interested in understanding how
segregation affects population characteristics and income disparity in
US cities. We will focus on two outcome variables: the poverty rate for
blacks and whites. Regress these two outcome variables on segregation in
1990, our explanatory variable, and interpret your results. Report
robust standard errors. Make sure you specify the units of measurement
in your
interpretation.}\label{question-we-are-interested-in-understanding-how-segregation-affects-population-characteristics-and-income-disparity-in-us-cities.-we-will-focus-on-two-outcome-variables-the-poverty-rate-for-blacks-and-whites.-regress-these-two-outcome-variables-on-segregation-in-1990-our-explanatory-variable-and-interpret-your-results.-report-robust-standard-errors.-make-sure-you-specify-the-units-of-measurement-in-your-interpretation.}}

\textbf{Code:}

\begin{Shaded}
\begin{Highlighting}[]
\CommentTok{\#install.packages("lmtest", "sandwich")}
\FunctionTok{library}\NormalTok{(lmtest)}
\end{Highlighting}
\end{Shaded}

\begin{verbatim}
## Warning: 程辑包'lmtest'是用R版本4.3.2 来建造的
\end{verbatim}

\begin{verbatim}
## 载入需要的程辑包:zoo
\end{verbatim}

\begin{verbatim}
## 
## 载入程辑包:'zoo'
\end{verbatim}

\begin{verbatim}
## The following objects are masked from 'package:base':
## 
##     as.Date, as.Date.numeric
\end{verbatim}

\begin{verbatim}
## 
## 载入程辑包:'lmtest'
\end{verbatim}

\begin{verbatim}
## The following object is masked from 'package:lfe':
## 
##     waldtest
\end{verbatim}

\begin{Shaded}
\begin{Highlighting}[]
\FunctionTok{library}\NormalTok{(sandwich)}
\end{Highlighting}
\end{Shaded}

\begin{Shaded}
\begin{Highlighting}[]
\CommentTok{\#Regress the two outcome variables on segregation in 1990}

\NormalTok{model.black }\OtherTok{\textless{}{-}} \FunctionTok{lm}\NormalTok{(povrate\_b }\SpecialCharTok{\textasciitilde{}}\NormalTok{ dism1990, }\AttributeTok{data =}\NormalTok{ data)}
\NormalTok{model.white }\OtherTok{\textless{}{-}} \FunctionTok{lm}\NormalTok{(povrate\_w }\SpecialCharTok{\textasciitilde{}}\NormalTok{ dism1990, }\AttributeTok{data =}\NormalTok{ data)}

\FunctionTok{stargazer}\NormalTok{(model.black, model.white, }\AttributeTok{header =} \ConstantTok{FALSE}\NormalTok{, }\AttributeTok{type =} \StringTok{"text"}\NormalTok{, }\AttributeTok{se =} \FunctionTok{list}\NormalTok{(model.black}\SpecialCharTok{$}\NormalTok{rse, model.white}\SpecialCharTok{$}\NormalTok{rse))}
\end{Highlighting}
\end{Shaded}

\begin{verbatim}
## 
## ===========================================================
##                                    Dependent variable:     
##                                ----------------------------
##                                  povrate_b      povrate_w  
##                                     (1)            (2)     
## -----------------------------------------------------------
## dism1990                          0.182***      -0.073***  
##                                   (0.051)        (0.022)   
##                                                            
## Constant                          0.161***      0.136***   
##                                   (0.030)        (0.013)   
##                                                            
## -----------------------------------------------------------
## Observations                        121            121     
## R2                                 0.095          0.081    
## Adjusted R2                        0.088          0.074    
## Residual Std. Error (df = 119)     0.076          0.033    
## F Statistic (df = 1; 119)        12.511***      10.538***  
## ===========================================================
## Note:                           *p<0.1; **p<0.05; ***p<0.01
\end{verbatim}

\begin{Shaded}
\begin{Highlighting}[]
\CommentTok{\# Report robust standard errors}

\NormalTok{robust\_black }\OtherTok{\textless{}{-}} \FunctionTok{coeftest}\NormalTok{(model.black, }\AttributeTok{vcov =}\NormalTok{ vcovHC)}
\NormalTok{robust\_white }\OtherTok{\textless{}{-}} \FunctionTok{coeftest}\NormalTok{(model.white, }\AttributeTok{vcov =}\NormalTok{ vcovHC)}

\FunctionTok{stargazer}\NormalTok{(robust\_black, robust\_white, }\AttributeTok{header =} \ConstantTok{FALSE}\NormalTok{, }\AttributeTok{type =} \StringTok{"text"}\NormalTok{)}
\end{Highlighting}
\end{Shaded}

\begin{verbatim}
## 
## =====================================
##              Dependent variable:     
##          ----------------------------
##                                      
##               (1)            (2)     
## -------------------------------------
## dism1990    0.182***      -0.073***  
##             (0.046)        (0.020)   
##                                      
## Constant    0.161***      0.136***   
##             (0.029)        (0.012)   
##                                      
## =====================================
## =====================================
## Note:     *p<0.1; **p<0.05; ***p<0.01
\end{verbatim}

\textbf{Answer:} The segregation in 1990 and the poverty rate for blacks
is positive related, which means with 1 unit increasing of segregation,
the poverty rate for blacks is expected to increase by 0.182 percentage
points. The segregation in 1990 and the poverty rate for whites is
negative related, which means with 1 unit increasing of segregation, the
poverty rate for whites is expected to decrease by 0.073 percentage
points.

\clearpage

\hypertarget{question-explain-the-problem-with-giving-a-causal-interpretation-to-the-estimates-you-just-produced.-give-examples-of-specific-factors-that-might-make-a-causal-interpretation-of-your-result-problematic.}{%
\subsection{6.2 Question: Explain the problem with giving a causal
interpretation to the estimates you just produced. Give examples of
specific factors that might make a causal interpretation of your result
problematic.}\label{question-explain-the-problem-with-giving-a-causal-interpretation-to-the-estimates-you-just-produced.-give-examples-of-specific-factors-that-might-make-a-causal-interpretation-of-your-result-problematic.}}

\textbf{Answer:}

\clearpage

\hypertarget{validity-of-the-instrument}{%
\section{7. Validity of the
instrument:}\label{validity-of-the-instrument}}

\hypertarget{question-estimate-the-following-regression-and-interpret-its-coefficients}{%
\subsection{7.1 Question: Estimate the following regression and
interpret it's
coefficients,}\label{question-estimate-the-following-regression-and-interpret-its-coefficients}}

\[
 dism1990_i=\beta_0+\beta_1RDI_i+\beta_2 tracklength_i+\epsilon.
\]

\textbf{Code:}

\begin{Shaded}
\begin{Highlighting}[]
\NormalTok{reg.dism }\OtherTok{\textless{}{-}} \FunctionTok{felm}\NormalTok{(dism1990 }\SpecialCharTok{\textasciitilde{}}\NormalTok{ herf }\SpecialCharTok{+}\NormalTok{ lenper, data)}
\FunctionTok{stargazer}\NormalTok{(reg.dism, }\AttributeTok{header =} \ConstantTok{FALSE}\NormalTok{, }\AttributeTok{type =} \StringTok{"text"}\NormalTok{)}
\end{Highlighting}
\end{Shaded}

\begin{verbatim}
## 
## ===============================================
##                         Dependent variable:    
##                     ---------------------------
##                              dism1990          
## -----------------------------------------------
## herf                         0.357***          
##                               (0.081)          
##                                                
## lenper                       18.514**          
##                               (9.126)          
##                                                
## Constant                     0.294***          
##                               (0.058)          
##                                                
## -----------------------------------------------
## Observations                    121            
## R2                             0.203           
## Adjusted R2                    0.189           
## Residual Std. Error      0.122 (df = 118)      
## ===============================================
## Note:               *p<0.1; **p<0.05; ***p<0.01
\end{verbatim}

\textbf{Answer:}

\clearpage

\hypertarget{question-re-estimate-the-specification-above-using-the-scale-command-around-the-variables-you-wish-to-standardize-in-the-regression.-what-do-you-notice}{%
\subsection{\texorpdfstring{7.2 Question: Re-estimate the specification
above using the \texttt{scale()} command around the variables you wish
to standardize in the regression. What do you
notice?}{7.2 Question: Re-estimate the specification above using the scale() command around the variables you wish to standardize in the regression. What do you notice?}}\label{question-re-estimate-the-specification-above-using-the-scale-command-around-the-variables-you-wish-to-standardize-in-the-regression.-what-do-you-notice}}

\textbf{Code:}

\begin{Shaded}
\begin{Highlighting}[]
\NormalTok{reg.sd }\OtherTok{\textless{}{-}} \FunctionTok{felm}\NormalTok{(dism1990 }\SpecialCharTok{\textasciitilde{}}\NormalTok{ herf }\SpecialCharTok{+} \FunctionTok{scale}\NormalTok{(lenper), data)}
\FunctionTok{stargazer}\NormalTok{(reg.sd, }\AttributeTok{header =} \ConstantTok{FALSE}\NormalTok{, }\AttributeTok{type =} \StringTok{"text"}\NormalTok{)}
\end{Highlighting}
\end{Shaded}

\begin{verbatim}
## 
## ===============================================
##                         Dependent variable:    
##                     ---------------------------
##                              dism1990          
## -----------------------------------------------
## herf                         0.357***          
##                               (0.081)          
##                                                
## scale(lenper)                 0.023**          
##                               (0.012)          
##                                                
## Constant                     0.310***          
##                               (0.060)          
##                                                
## -----------------------------------------------
## Observations                    121            
## R2                             0.203           
## Adjusted R2                    0.189           
## Residual Std. Error      0.122 (df = 118)      
## ===============================================
## Note:               *p<0.1; **p<0.05; ***p<0.01
\end{verbatim}

\textbf{Answer:}

\clearpage

\hypertarget{question-in-the-context-of-instrumental-variables-what-is-this-regression-referred-to-as-and-why-is-it-important}{%
\subsection{7.3 Question: In the context of instrumental variables, what
is this regression referred to as and why is it
important?}\label{question-in-the-context-of-instrumental-variables-what-is-this-regression-referred-to-as-and-why-is-it-important}}

\textbf{Answer:} This referred to first stage explantery power explain
in some way to segregation, RDI to be a good instrument

\clearpage

\hypertarget{question-illustrate-the-relationship-between-the-rdi-and-segregation-graphically.}{%
\subsection{7.4 Question: Illustrate the relationship between the RDI
and segregation
graphically.}\label{question-illustrate-the-relationship-between-the-rdi-and-segregation-graphically.}}

\textbf{Code:}

\begin{Shaded}
\begin{Highlighting}[]
\FunctionTok{library}\NormalTok{(ggplot2)}
\end{Highlighting}
\end{Shaded}

\begin{verbatim}
## Warning: 程辑包'ggplot2'是用R版本4.3.2 来建造的
\end{verbatim}

\begin{Shaded}
\begin{Highlighting}[]
\NormalTok{plotted }\OtherTok{\textless{}{-}} \FunctionTok{ggplot}\NormalTok{(}\AttributeTok{data =}\NormalTok{ data, }\FunctionTok{aes}\NormalTok{(}\AttributeTok{x =}\NormalTok{ herf, }\AttributeTok{y =}\NormalTok{ dism1990, }\AttributeTok{color =}\NormalTok{ variable)) }\SpecialCharTok{+}
  \FunctionTok{geom\_point}\NormalTok{(}\FunctionTok{aes}\NormalTok{(}\AttributeTok{y =}\NormalTok{ dism1990), }\AttributeTok{color =} \StringTok{"gold"}\NormalTok{) }\SpecialCharTok{+}
  \FunctionTok{geom\_smooth}\NormalTok{(}\AttributeTok{method =} \StringTok{"lm"}\NormalTok{, }\FunctionTok{aes}\NormalTok{(}\AttributeTok{y =}\NormalTok{ dism1990, }\AttributeTok{col =} \StringTok{"dism1990"}\NormalTok{), }\AttributeTok{se =}\NormalTok{ F, }\AttributeTok{color =} \StringTok{"orange"}\NormalTok{) }\SpecialCharTok{+}
  \FunctionTok{theme\_bw}\NormalTok{() }\SpecialCharTok{+}
  \FunctionTok{labs}\NormalTok{(}\AttributeTok{title =} \StringTok{"The Relationship Between the RDI and Segregation"}\NormalTok{, }\AttributeTok{x =} \StringTok{"RDI"}\NormalTok{, }\AttributeTok{y =} \StringTok{"dism1990"}\NormalTok{, }\AttributeTok{color =} \StringTok{"Regressand Variable"}\NormalTok{)}
 
\NormalTok{plotted}
\end{Highlighting}
\end{Shaded}

\begin{verbatim}
## `geom_smooth()` using formula = 'y ~ x'
\end{verbatim}

\includegraphics{PS3_IV_files/figure-latex/unnamed-chunk-10-1.pdf}

\clearpage

\hypertarget{question-is-there-a-concern-that-this-might-be-a-weak-instrument-why-would-this-be-a-problem}{%
\subsection{7.5 Question: Is there a concern that this might be a weak
instrument? Why would this be a
problem?}\label{question-is-there-a-concern-that-this-might-be-a-weak-instrument-why-would-this-be-a-problem}}

\textbf{Answer:}

\begin{Shaded}
\begin{Highlighting}[]
\FunctionTok{summary}\NormalTok{(reg.sd)}
\end{Highlighting}
\end{Shaded}

\begin{verbatim}
## 
## Call:
##    felm(formula = dism1990 ~ herf + scale(lenper), data = data) 
## 
## Residuals:
##      Min       1Q   Median       3Q      Max 
## -0.22582 -0.10817  0.00864  0.08031  0.32113 
## 
## Coefficients:
##               Estimate Std. Error t value Pr(>|t|)    
## (Intercept)    0.31025    0.05984   5.185 9.05e-07 ***
## herf           0.35731    0.08130   4.395 2.43e-05 ***
## scale(lenper)  0.02333    0.01150   2.029   0.0447 *  
## ---
## Signif. codes:  0 '***' 0.001 '**' 0.01 '*' 0.05 '.' 0.1 ' ' 1
## 
## Residual standard error: 0.1218 on 118 degrees of freedom
## Multiple R-squared(full model): 0.2025   Adjusted R-squared: 0.189 
## Multiple R-squared(proj model): 0.2025   Adjusted R-squared: 0.189 
## F-statistic(full model):14.98 on 2 and 118 DF, p-value: 1.591e-06 
## F-statistic(proj model): 14.98 on 2 and 118 DF, p-value: 1.591e-06
\end{verbatim}

f stasistic compare to 10 greater not weak

\clearpage

\hypertarget{question-select-a-number-of-relevant-city-characteristics-in-the-data-to-regress-on-the-rdi-and-track-length.-present-your-results-and-interpret-your-findings.-why-do-these-results-matter-for-answering-our-question-of-interest}{%
\subsection{7.6 Question: Select a number of relevant city
characteristics in the data to regress on the RDI and track length.
Present your results and interpret your findings. Why do these results
matter for answering our question of
interest?}\label{question-select-a-number-of-relevant-city-characteristics-in-the-data-to-regress-on-the-rdi-and-track-length.-present-your-results-and-interpret-your-findings.-why-do-these-results-matter-for-answering-our-question-of-interest}}

\textbf{Code and Answer:}

\begin{Shaded}
\begin{Highlighting}[]
\NormalTok{reg}\FloatTok{.1} \OtherTok{\textless{}{-}} \FunctionTok{felm}\NormalTok{(area1910 }\SpecialCharTok{\textasciitilde{}}\NormalTok{ herf }\SpecialCharTok{+}\NormalTok{ lenper, data)}
\NormalTok{reg}\FloatTok{.2} \OtherTok{\textless{}{-}} \FunctionTok{felm}\NormalTok{(count1910 }\SpecialCharTok{\textasciitilde{}}\NormalTok{ herf }\SpecialCharTok{+}\NormalTok{ lenper, data)}
\NormalTok{reg}\FloatTok{.3} \OtherTok{\textless{}{-}} \FunctionTok{felm}\NormalTok{(black1910 }\SpecialCharTok{\textasciitilde{}}\NormalTok{ herf }\SpecialCharTok{+}\NormalTok{ lenper, data)}
\NormalTok{reg}\FloatTok{.4} \OtherTok{\textless{}{-}} \FunctionTok{felm}\NormalTok{(ethseg10 }\SpecialCharTok{\textasciitilde{}}\NormalTok{ herf }\SpecialCharTok{+}\NormalTok{ lenper, data)}
\NormalTok{reg}\FloatTok{.5} \OtherTok{\textless{}{-}} \FunctionTok{felm}\NormalTok{(ethiso10 }\SpecialCharTok{\textasciitilde{}}\NormalTok{ herf }\SpecialCharTok{+}\NormalTok{ lenper, data)}


\FunctionTok{stargazer}\NormalTok{(reg}\FloatTok{.1}\NormalTok{, reg}\FloatTok{.2}\NormalTok{, reg}\FloatTok{.3}\NormalTok{, reg}\FloatTok{.4}\NormalTok{, reg}\FloatTok{.5}\NormalTok{, }\AttributeTok{header =} \ConstantTok{FALSE}\NormalTok{, }\AttributeTok{type =} \StringTok{"text"}\NormalTok{, }\AttributeTok{title =} \StringTok{"impact of city characteristics on RDI"}\NormalTok{, }\AttributeTok{se =} \FunctionTok{list}\NormalTok{(reg}\FloatTok{.1}\SpecialCharTok{$}\NormalTok{rse, reg}\FloatTok{.2}\SpecialCharTok{$}\NormalTok{rse, reg}\FloatTok{.4}\SpecialCharTok{$}\NormalTok{rse, reg}\FloatTok{.5}\SpecialCharTok{$}\NormalTok{rse))}
\end{Highlighting}
\end{Shaded}

\begin{verbatim}
## 
## impact of city characteristics on RDI
## ==============================================================================================================
##                                                        Dependent variable:                                    
##                     ------------------------------------------------------------------------------------------
##                           area1910            count1910          black1910        ethseg10        ethiso10    
##                             (1)                  (2)                (3)              (4)             (5)      
## --------------------------------------------------------------------------------------------------------------
## herf                     -3,992.637            665.751             -0.001           0.076           0.027     
##                         (11,986.490)         (1,362.964)          (0.185)          (0.070)         (0.086)    
##                                                                                                               
## lenper                  -574,401.000          75,553.190           9.236           15.343          -12.439    
##                        (553,669.000)        (134,814.900)         (53.248)        (17.288)        (19.930)    
##                                                                                                               
## Constant                18,409.570**           976.876             0.007          0.238***          0.048     
##                         (8,612.320)           (927.189)           (0.121)          (0.051)         (0.054)    
##                                                                                                               
## --------------------------------------------------------------------------------------------------------------
## Observations                 58                  121                121              49              49       
## R2                         0.007                0.006              0.290            0.014           0.009     
## Adjusted R2                -0.029               -0.011             0.278           -0.029          -0.034     
## Residual Std. Error 15,050.340 (df = 55) 1,903.415 (df = 118) 0.018 (df = 118) 0.184 (df = 46) 0.075 (df = 46)
## ==============================================================================================================
## Note:                                                                              *p<0.1; **p<0.05; ***p<0.01
\end{verbatim}

resctriction collusion hold

\clearpage

\hypertarget{question-what-are-the-two-conditions-necessary-for-a-valid-instrument-what-evidence-do-you-have-that-the-rdi-meet-these-conditions-be-specific-in-supporting-this-claim.}{%
\subsection{7.7 Question: What are the two conditions necessary for a
valid instrument? What evidence do you have that the RDI meet these
conditions? Be specific in supporting this
claim.}\label{question-what-are-the-two-conditions-necessary-for-a-valid-instrument-what-evidence-do-you-have-that-the-rdi-meet-these-conditions-be-specific-in-supporting-this-claim.}}

\textbf{Answer:} Cov(z, x\_1) != 0 (the first stage) Cov(z, v) = 0 (the
exclusion restriction) y:poverty rate x\_1: dism1990(segregation) z:
herf(RDI) v: city characteristics

first stage quite strong 很显著 in paper why RDI made the conditions

\clearpage

\hypertarget{question-do-you-believe-the-instrument-is-valid-whywhy-not}{%
\subsection{7.8 Question: Do you believe the instrument is valid?
Why/why
not?}\label{question-do-you-believe-the-instrument-is-valid-whywhy-not}}

\textbf{Answer:} something elso 也许有其他的可以影响,RDI
也许通过其他因素影响 poverty rate

\clearpage

\hypertarget{question-generate-a-table-that-estimates-the-effect-of-segregation-on-the-poverty-rate-for-blacks-and-whites-by-ols-and-then-using-the-rdi-instrument.-make-sure-you-report-robust-standard-errors.-how-does-the-use-of-the-rdi-instrument-change-the-estimated-coefficients}{%
\subsection{7.9 Question: Generate a table that estimates the effect of
segregation on the poverty rate for blacks and whites by OLS and then
using the RDI instrument. Make sure you report robust standard errors.
How does the use of the RDI instrument change the estimated
coefficients?}\label{question-generate-a-table-that-estimates-the-effect-of-segregation-on-the-poverty-rate-for-blacks-and-whites-by-ols-and-then-using-the-rdi-instrument.-make-sure-you-report-robust-standard-errors.-how-does-the-use-of-the-rdi-instrument-change-the-estimated-coefficients}}

\textbf{Code and Answer:}

\begin{Shaded}
\begin{Highlighting}[]
\NormalTok{model.black.ols }\OtherTok{\textless{}{-}} \FunctionTok{felm}\NormalTok{(povrate\_b }\SpecialCharTok{\textasciitilde{}}\NormalTok{ dism1990, }\AttributeTok{data =}\NormalTok{ data)}
\NormalTok{model.white.ols }\OtherTok{\textless{}{-}} \FunctionTok{felm}\NormalTok{(povrate\_w }\SpecialCharTok{\textasciitilde{}}\NormalTok{ dism1990, }\AttributeTok{data =}\NormalTok{ data)}

\NormalTok{model.black.RDI }\OtherTok{\textless{}{-}} \FunctionTok{felm}\NormalTok{(povrate\_b }\SpecialCharTok{\textasciitilde{}}\NormalTok{ lenper}\SpecialCharTok{|}\DecValTok{0}\SpecialCharTok{|}\NormalTok{(dism1990 }\SpecialCharTok{\textasciitilde{}}\NormalTok{ herf), }\AttributeTok{data =}\NormalTok{ data)}
\NormalTok{model.white.RDI }\OtherTok{\textless{}{-}} \FunctionTok{felm}\NormalTok{(povrate\_w }\SpecialCharTok{\textasciitilde{}}\NormalTok{ lenper}\SpecialCharTok{|}\DecValTok{0}\SpecialCharTok{|}\NormalTok{(dism1990 }\SpecialCharTok{\textasciitilde{}}\NormalTok{ herf), }\AttributeTok{data =}\NormalTok{ data)}

\FunctionTok{stargazer}\NormalTok{(model.black.ols, model.white.ols, model.black.RDI, model.white.RDI, }\AttributeTok{header =} \ConstantTok{FALSE}\NormalTok{, }\AttributeTok{type =} \StringTok{"text"}\NormalTok{, }\AttributeTok{se =} \FunctionTok{list}\NormalTok{(model.black.ols}\SpecialCharTok{$}\NormalTok{rse, model.white.ols}\SpecialCharTok{$}\NormalTok{rse,model.black.RDI}\SpecialCharTok{$}\NormalTok{rse, model.white.RDI}\SpecialCharTok{$}\NormalTok{rse))}
\end{Highlighting}
\end{Shaded}

\begin{verbatim}
## 
## =======================================================================================
##                                             Dependent variable:                        
##                     -------------------------------------------------------------------
##                        povrate_b        povrate_w        povrate_b        povrate_w    
##                           (1)              (2)              (3)              (4)       
## ---------------------------------------------------------------------------------------
## dism1990                0.182***        -0.073***                                      
##                         (0.045)          (0.019)                                       
##                                                                                        
## lenper                                                     -4.780           0.602      
##                                                           (3.067)          (1.970)     
##                                                                                        
## `dism1990(fit)`                                           0.258**         -0.196***    
##                                                           (0.108)          (0.065)     
##                                                                                        
## Constant                0.161***         0.136***         0.121**          0.205***    
##                         (0.029)          (0.012)          (0.061)          (0.037)     
##                                                                                        
## ---------------------------------------------------------------------------------------
## Observations              121              121              121              121       
## R2                       0.095            0.081            0.084            -0.150     
## Adjusted R2              0.088            0.074            0.068            -0.170     
## Residual Std. Error 0.076 (df = 119) 0.033 (df = 119) 0.077 (df = 118) 0.037 (df = 118)
## =======================================================================================
## Note:                                                       *p<0.1; **p<0.05; ***p<0.01
\end{verbatim}

\clearpage

\hypertarget{question-what-is-the-reduced-form-equation}{%
\subsection{7.10 Question: What is the reduced form
equation?}\label{question-what-is-the-reduced-form-equation}}

\textbf{Answer:} \[
povrateblack_i = \pi_i0 + \pi_1 \times herf_i + \eta\\
povratewhite_i = \gamma_i0 + \gamma_1 \times herf_i + \eta
\]

\clearpage

\hypertarget{question-2-pages-for-the-two-poverty-rates-estimate-the-reduced-form-on-all-the-cities-and-illustrate-the-reduced-form-relationships-graphically.}{%
\subsection{7.11 Question: (2 pages) For the two poverty rates, estimate
the reduced form on all the cities and illustrate the reduced form
relationships
graphically.}\label{question-2-pages-for-the-two-poverty-rates-estimate-the-reduced-form-on-all-the-cities-and-illustrate-the-reduced-form-relationships-graphically.}}

\textbf{Code:}

\begin{Shaded}
\begin{Highlighting}[]
\NormalTok{model.black.rf }\OtherTok{\textless{}{-}} \FunctionTok{felm}\NormalTok{(povrate\_b }\SpecialCharTok{\textasciitilde{}}\NormalTok{ herf, }\AttributeTok{data =}\NormalTok{ data)}
\NormalTok{model.white.rf }\OtherTok{\textless{}{-}} \FunctionTok{felm}\NormalTok{(povrate\_w }\SpecialCharTok{\textasciitilde{}}\NormalTok{ herf, }\AttributeTok{data =}\NormalTok{ data)}

\FunctionTok{stargazer}\NormalTok{(model.black.rf, model.white.rf, }\AttributeTok{header =} \ConstantTok{FALSE}\NormalTok{, }\AttributeTok{type =} \StringTok{"text"}\NormalTok{, }\AttributeTok{se =} \FunctionTok{list}\NormalTok{(model.black.rf}\SpecialCharTok{$}\NormalTok{rse, model.white.rf}\SpecialCharTok{$}\NormalTok{rse))}
\end{Highlighting}
\end{Shaded}

\begin{verbatim}
## 
## ===========================================================
##                                    Dependent variable:     
##                                ----------------------------
##                                  povrate_b      povrate_w  
##                                     (1)            (2)     
## -----------------------------------------------------------
## herf                              0.092**       -0.077***  
##                                   (0.046)        (0.022)   
##                                                            
## Constant                          0.197***      0.150***   
##                                   (0.036)        (0.017)   
##                                                            
## -----------------------------------------------------------
## Observations                        121            121     
## R2                                 0.027          0.099    
## Adjusted R2                        0.019          0.092    
## Residual Std. Error (df = 119)     0.079          0.033    
## ===========================================================
## Note:                           *p<0.1; **p<0.05; ***p<0.01
\end{verbatim}

\begin{Shaded}
\begin{Highlighting}[]
\NormalTok{plotted }\OtherTok{\textless{}{-}} \FunctionTok{ggplot}\NormalTok{(}\AttributeTok{data =}\NormalTok{ data, }\FunctionTok{aes}\NormalTok{(}\AttributeTok{x =}\NormalTok{ herf, }\AttributeTok{y =}\NormalTok{ dism1990, }\AttributeTok{color =}\NormalTok{ variable)) }\SpecialCharTok{+}
  \FunctionTok{geom\_point}\NormalTok{(}\FunctionTok{aes}\NormalTok{(}\AttributeTok{y =}\NormalTok{ dism1990), }\AttributeTok{color =} \StringTok{"gold"}\NormalTok{) }\SpecialCharTok{+}
  \FunctionTok{geom\_smooth}\NormalTok{(}\AttributeTok{method =} \StringTok{"lm"}\NormalTok{, }\FunctionTok{aes}\NormalTok{(}\AttributeTok{y =}\NormalTok{ dism1990, }\AttributeTok{col =} \StringTok{"dism1990"}\NormalTok{), }\AttributeTok{se =}\NormalTok{ F, }\AttributeTok{color =} \StringTok{"orange"}\NormalTok{) }\SpecialCharTok{+}
  \FunctionTok{theme\_bw}\NormalTok{() }\SpecialCharTok{+}
  \FunctionTok{labs}\NormalTok{(}\AttributeTok{title =} \StringTok{"The Relationship Between the RDI and Segregation"}\NormalTok{, }\AttributeTok{x =} \StringTok{"RDI"}\NormalTok{, }\AttributeTok{y =} \StringTok{"dism1990"}\NormalTok{, }\AttributeTok{color =} \StringTok{"Regressand Variable"}\NormalTok{)}
 
\NormalTok{plotted}
\end{Highlighting}
\end{Shaded}

\begin{verbatim}
## `geom_smooth()` using formula = 'y ~ x'
\end{verbatim}

\includegraphics{PS3_IV_files/figure-latex/unnamed-chunk-15-1.pdf}

\clearpage

\hypertarget{question-generate-a-table-with-at-least-six-estimations-that-checks-whether-the-main-results-are-robust-to-adding-additional-controls-for-city-characteristics.-what-do-you-conclude}{%
\subsection{7.12 Question: Generate a table with at least six
estimations that checks whether the main results are robust to adding
additional controls for city characteristics. What do you
conclude?}\label{question-generate-a-table-with-at-least-six-estimations-that-checks-whether-the-main-results-are-robust-to-adding-additional-controls-for-city-characteristics.-what-do-you-conclude}}

\textbf{Code:}

\begin{Shaded}
\begin{Highlighting}[]
\NormalTok{reg}\FloatTok{.71} \OtherTok{\textless{}{-}} \FunctionTok{felm}\NormalTok{(povrate\_w }\SpecialCharTok{\textasciitilde{}}\NormalTok{ lenper}\SpecialCharTok{|}\DecValTok{0}\SpecialCharTok{|}\NormalTok{(dism1990 }\SpecialCharTok{\textasciitilde{}}\NormalTok{ herf }\SpecialCharTok{+}\NormalTok{ lenper), data)}
\end{Highlighting}
\end{Shaded}

\begin{verbatim}
## Warning in chol.default(mat, pivot = TRUE, tol = tol): the matrix is either
## rank-deficient or not positive definite
\end{verbatim}

\begin{Shaded}
\begin{Highlighting}[]
\NormalTok{reg}\FloatTok{.72} \OtherTok{\textless{}{-}} \FunctionTok{felm}\NormalTok{(povrate\_w }\SpecialCharTok{\textasciitilde{}}\NormalTok{ lenper }\SpecialCharTok{+}\NormalTok{ pctbk1990}\SpecialCharTok{|}\DecValTok{0}\SpecialCharTok{|}\NormalTok{(dism1990 }\SpecialCharTok{\textasciitilde{}}\NormalTok{ herf }\SpecialCharTok{+}\NormalTok{lenper }\SpecialCharTok{+}\NormalTok{ pctbk1990),data)}
\end{Highlighting}
\end{Shaded}

\begin{verbatim}
## Warning in chol.default(mat, pivot = TRUE, tol = tol): the matrix is either
## rank-deficient or not positive definite
\end{verbatim}

\begin{Shaded}
\begin{Highlighting}[]
\NormalTok{reg}\FloatTok{.73} \OtherTok{\textless{}{-}} \FunctionTok{felm}\NormalTok{(povrate\_w }\SpecialCharTok{\textasciitilde{}}\NormalTok{ lenper }\SpecialCharTok{+}\NormalTok{ pctbk1990 }\SpecialCharTok{+}\NormalTok{ lfp1920}\SpecialCharTok{|}\DecValTok{0}\SpecialCharTok{|}\NormalTok{(dism1990 }\SpecialCharTok{\textasciitilde{}}\NormalTok{ herf }\SpecialCharTok{+}\NormalTok{lenper }\SpecialCharTok{+}\NormalTok{ pctbk1990 }\SpecialCharTok{+}\NormalTok{lfp1920),data)}
\end{Highlighting}
\end{Shaded}

\begin{verbatim}
## Warning in chol.default(mat, pivot = TRUE, tol = tol): the matrix is either
## rank-deficient or not positive definite
\end{verbatim}

\begin{Shaded}
\begin{Highlighting}[]
\NormalTok{reg}\FloatTok{.74} \OtherTok{\textless{}{-}} \FunctionTok{felm}\NormalTok{(povrate\_b }\SpecialCharTok{\textasciitilde{}}\NormalTok{ lenper}\SpecialCharTok{|}\DecValTok{0}\SpecialCharTok{|}\NormalTok{(dism1990 }\SpecialCharTok{\textasciitilde{}}\NormalTok{ herf }\SpecialCharTok{+}\NormalTok{ lenper), data)}
\end{Highlighting}
\end{Shaded}

\begin{verbatim}
## Warning in chol.default(mat, pivot = TRUE, tol = tol): the matrix is either
## rank-deficient or not positive definite
\end{verbatim}

\begin{Shaded}
\begin{Highlighting}[]
\NormalTok{reg}\FloatTok{.75} \OtherTok{\textless{}{-}} \FunctionTok{felm}\NormalTok{(povrate\_b }\SpecialCharTok{\textasciitilde{}}\NormalTok{ lenper }\SpecialCharTok{+}\NormalTok{ pctbk1990}\SpecialCharTok{|}\DecValTok{0}\SpecialCharTok{|}\NormalTok{(dism1990 }\SpecialCharTok{\textasciitilde{}}\NormalTok{ herf }\SpecialCharTok{+}\NormalTok{lenper }\SpecialCharTok{+}\NormalTok{ pctbk1990),data)}
\end{Highlighting}
\end{Shaded}

\begin{verbatim}
## Warning in chol.default(mat, pivot = TRUE, tol = tol): the matrix is either
## rank-deficient or not positive definite
\end{verbatim}

\begin{Shaded}
\begin{Highlighting}[]
\NormalTok{reg}\FloatTok{.76} \OtherTok{\textless{}{-}} \FunctionTok{felm}\NormalTok{(povrate\_b }\SpecialCharTok{\textasciitilde{}}\NormalTok{ lenper }\SpecialCharTok{+}\NormalTok{ pctbk1990 }\SpecialCharTok{+}\NormalTok{ lfp1920}\SpecialCharTok{|}\DecValTok{0}\SpecialCharTok{|}\NormalTok{(dism1990 }\SpecialCharTok{\textasciitilde{}}\NormalTok{ herf }\SpecialCharTok{+}\NormalTok{lenper }\SpecialCharTok{+}\NormalTok{ pctbk1990 }\SpecialCharTok{+}\NormalTok{lfp1920),data)}
\end{Highlighting}
\end{Shaded}

\begin{verbatim}
## Warning in chol.default(mat, pivot = TRUE, tol = tol): the matrix is either
## rank-deficient or not positive definite
\end{verbatim}

\begin{Shaded}
\begin{Highlighting}[]
\FunctionTok{stargazer}\NormalTok{(reg}\FloatTok{.71}\NormalTok{, reg}\FloatTok{.72}\NormalTok{, reg}\FloatTok{.73}\NormalTok{,reg}\FloatTok{.74}\NormalTok{, reg}\FloatTok{.75}\NormalTok{, reg}\FloatTok{.76}\NormalTok{, }\AttributeTok{header =} \ConstantTok{FALSE}\NormalTok{, }\AttributeTok{type =} \StringTok{"text"}\NormalTok{, }\AttributeTok{se =} \FunctionTok{list}\NormalTok{(reg}\FloatTok{.71}\SpecialCharTok{$}\NormalTok{rse, reg}\FloatTok{.72}\SpecialCharTok{$}\NormalTok{rse, reg}\FloatTok{.73}\SpecialCharTok{$}\NormalTok{rse,reg}\FloatTok{.74}\SpecialCharTok{$}\NormalTok{rse, reg}\FloatTok{.75}\SpecialCharTok{$}\NormalTok{rse, reg}\FloatTok{.76}\SpecialCharTok{$}\NormalTok{rse))}
\end{Highlighting}
\end{Shaded}

\begin{verbatim}
## 
## =========================================================================================================================
##                                                              Dependent variable:                                         
##                     -----------------------------------------------------------------------------------------------------
##                                         povrate_w                                          povrate_b                     
##                           (1)              (2)              (3)              (4)              (5)              (6)       
## -------------------------------------------------------------------------------------------------------------------------
## lenper                   0.602            -0.479           -0.918           -4.780           -2.331           -1.586     
##                         (1.970)          (1.801)          (1.550)          (3.067)          (2.402)          (2.655)     
##                                                                                                                          
## pctbk1990                                 0.211            0.180                            -0.478*          -0.426*     
##                                          (0.153)          (0.136)                           (0.246)          (0.230)     
##                                                                                                                          
## lfp1920                                                    -0.122                                             0.207      
##                                                           (0.105)                                            (0.208)     
##                                                                                                                          
## `dism1990(fit)`        -0.196***         -0.241**         -0.224**         0.258**          0.360**          0.333**     
##                         (0.065)          (0.097)          (0.088)          (0.108)          (0.141)          (0.132)     
##                                                                                                                          
## Constant                0.205***         0.219***         0.263***         0.121**           0.091            0.016      
##                         (0.037)          (0.048)          (0.073)          (0.061)          (0.068)          (0.120)     
##                                                                                                                          
## -------------------------------------------------------------------------------------------------------------------------
## Observations              121              121              121              121              121              121       
## R2                       -0.150           -0.254           -0.172           0.084            0.108            0.130      
## Adjusted R2              -0.170           -0.286           -0.212           0.068            0.085            0.100      
## Residual Std. Error 0.037 (df = 118) 0.039 (df = 117) 0.038 (df = 116) 0.077 (df = 118) 0.076 (df = 117) 0.076 (df = 116)
## =========================================================================================================================
## Note:                                                                                         *p<0.1; **p<0.05; ***p<0.01
\end{verbatim}

regression 选最有影响的一个city chara one more control variable, two
more, three more\ldots{} example of additional control variable :
percentage of black

\textbf{Answer:} controls do not have a large effect on the results

\clearpage

\hypertarget{why-two-stage-least-squares}{%
\section{\texorpdfstring{8. Why \textbf{Two Stage} least
squares?}{8. Why Two Stage least squares?}}\label{why-two-stage-least-squares}}

Because the estimates in this paper only feature one endogenous
regressor and one instrument, it is an excellent example with which to
illustrate build intuition and see what the instrumental variables
regressor is actually doing because in this scenario the IV estimator is
exactly equal to the two stage least squares estimator
(\(\hat{\beta}_{IV}=\hat{\beta}_{2SLS}\)).

\hypertarget{question-estimate-the-first-stage-regression-and-use-your-estimates-to-generate-the-predicted-values-for-the-explanatory-variable-for-all-the-observations.}{%
\subsection{8.1 Question: Estimate the first stage regression and use
your estimates to generate the predicted values for the explanatory
variable for all the
observations.}\label{question-estimate-the-first-stage-regression-and-use-your-estimates-to-generate-the-predicted-values-for-the-explanatory-variable-for-all-the-observations.}}

\textbf{Code:}

\begin{Shaded}
\begin{Highlighting}[]
\NormalTok{first\_stage }\OtherTok{\textless{}{-}} \FunctionTok{lm}\NormalTok{(dism1990 }\SpecialCharTok{\textasciitilde{}}\NormalTok{ herf, }\AttributeTok{data =}\NormalTok{ data)}
\FunctionTok{stargazer}\NormalTok{(first\_stage, }\AttributeTok{header =} \ConstantTok{FALSE}\NormalTok{, }\AttributeTok{type =} \StringTok{"text"}\NormalTok{, }\AttributeTok{se =} \FunctionTok{list}\NormalTok{(first\_stage}\SpecialCharTok{$}\NormalTok{rse))}
\end{Highlighting}
\end{Shaded}

\begin{verbatim}
## 
## ===============================================
##                         Dependent variable:    
##                     ---------------------------
##                              dism1990          
## -----------------------------------------------
## herf                         0.400***          
##                               (0.080)          
##                                                
## Constant                     0.280***          
##                               (0.059)          
##                                                
## -----------------------------------------------
## Observations                    121            
## R2                             0.175           
## Adjusted R2                    0.168           
## Residual Std. Error      0.123 (df = 119)      
## F Statistic           25.190*** (df = 1; 119)  
## ===============================================
## Note:               *p<0.1; **p<0.05; ***p<0.01
\end{verbatim}

\begin{Shaded}
\begin{Highlighting}[]
\NormalTok{data}\SpecialCharTok{$}\NormalTok{predicted\_dism1990 }\OtherTok{\textless{}{-}} \FunctionTok{predict}\NormalTok{(first\_stage)}
\end{Highlighting}
\end{Shaded}

\clearpage

\hypertarget{question-if-our-instrument-is-valid-the-step-above-removed-the-bad-endogenous-variation-from-the-predicted-explanatory-variable-keeping-only-the-exogenous-variation-that-is-generated-by-the-instrument.-now-run-the-second-stage-by-regressing-our-outcome-variable-on-the-predicted-values-generated-above-and-the-relevant-controls.-compare-your-estimates-from-this-regression-to-those-generated-earlier.-how-do-they-compare}{%
\subsection{8.2 Question: If our instrument is valid, the step above
``removed'' the ``bad'' endogenous variation from the predicted
explanatory variable, keeping only the exogenous variation that is
generated by the instrument. Now run the second stage by regressing our
outcome variable on the predicted values generated above and the
relevant controls. Compare your estimates from this regression to those
generated earlier. How do they
compare?}\label{question-if-our-instrument-is-valid-the-step-above-removed-the-bad-endogenous-variation-from-the-predicted-explanatory-variable-keeping-only-the-exogenous-variation-that-is-generated-by-the-instrument.-now-run-the-second-stage-by-regressing-our-outcome-variable-on-the-predicted-values-generated-above-and-the-relevant-controls.-compare-your-estimates-from-this-regression-to-those-generated-earlier.-how-do-they-compare}}

\textbf{Code:}

\begin{Shaded}
\begin{Highlighting}[]
\NormalTok{model.black.pd }\OtherTok{\textless{}{-}} \FunctionTok{felm}\NormalTok{(povrate\_b }\SpecialCharTok{\textasciitilde{}}\NormalTok{ predicted\_dism1990}\SpecialCharTok{+}\NormalTok{lenper, }\AttributeTok{data =}\NormalTok{ data)}
\NormalTok{model.white.pd }\OtherTok{\textless{}{-}} \FunctionTok{felm}\NormalTok{(povrate\_w }\SpecialCharTok{\textasciitilde{}}\NormalTok{ predicted\_dism1990}\SpecialCharTok{+}\NormalTok{lenper, }\AttributeTok{data =}\NormalTok{ data)}

\FunctionTok{stargazer}\NormalTok{(model.black.pd, model.white.pd, }\AttributeTok{header =} \ConstantTok{FALSE}\NormalTok{, }\AttributeTok{type =} \StringTok{"text"}\NormalTok{, }\AttributeTok{se =} \FunctionTok{list}\NormalTok{(model.black.pd}\SpecialCharTok{$}\NormalTok{rse, model.white.pd}\SpecialCharTok{$}\NormalTok{rse))}
\end{Highlighting}
\end{Shaded}

\begin{verbatim}
## 
## ===========================================================
##                                    Dependent variable:     
##                                ----------------------------
##                                  povrate_b      povrate_w  
##                                     (1)            (2)     
## -----------------------------------------------------------
## predicted_dism1990                 0.231*       -0.175***  
##                                   (0.120)        (0.054)   
##                                                            
## lenper                             0.004        -3.022***  
##                                   (4.398)        (1.011)   
##                                                            
## Constant                           0.133*       0.197***   
##                                   (0.069)        (0.032)   
##                                                            
## -----------------------------------------------------------
## Observations                        121            121     
## R2                                 0.027          0.111    
## Adjusted R2                        0.010          0.096    
## Residual Std. Error (df = 118)     0.079          0.033    
## ===========================================================
## Note:                           *p<0.1; **p<0.05; ***p<0.01
\end{verbatim}

\textbf{Answer:} pretty similar

\clearpage

\hypertarget{yet-another-iv-trick-taking-the-good-variation-and-scaling-it}{%
\section{9. Yet another IV trick: Taking the ``Good'' variation and
scaling
it}\label{yet-another-iv-trick-taking-the-good-variation-and-scaling-it}}

\hypertarget{question-take-the-coefficient-from-you-reduced-form-estimate-and-divide-it-by-your-first-stage-estimate.-how-does-this-value-compare-your-earlier-estimate-for-the-main-result}{%
\subsection{9.1 Question: Take the coefficient from you reduced form
estimate and divide it by your first stage estimate. How does this value
compare your earlier estimate for the main
result?}\label{question-take-the-coefficient-from-you-reduced-form-estimate-and-divide-it-by-your-first-stage-estimate.-how-does-this-value-compare-your-earlier-estimate-for-the-main-result}}

\textbf{Answer:}

\hypertarget{submission-instructions}{%
\section{10. Submission instructions:}\label{submission-instructions}}

\begin{itemize}
\tightlist
\item
  Make sure the final version of your assignment is knit in pdf format
  and uploaded to gradescope. Make sure you have one question response
  per page (unless otherwise indicated) so that question positions align
  with the template in gradescope.The final PDF should be 22 pages long.
\end{itemize}

\end{document}
