% Options for packages loaded elsewhere
\PassOptionsToPackage{unicode}{hyperref}
\PassOptionsToPackage{hyphens}{url}
%
\documentclass[
]{article}
\usepackage{amsmath,amssymb}
\usepackage{iftex}
\ifPDFTeX
  \usepackage[T1]{fontenc}
  \usepackage[utf8]{inputenc}
  \usepackage{textcomp} % provide euro and other symbols
\else % if luatex or xetex
  \usepackage{unicode-math} % this also loads fontspec
  \defaultfontfeatures{Scale=MatchLowercase}
  \defaultfontfeatures[\rmfamily]{Ligatures=TeX,Scale=1}
\fi
\usepackage{lmodern}
\ifPDFTeX\else
  % xetex/luatex font selection
\fi
% Use upquote if available, for straight quotes in verbatim environments
\IfFileExists{upquote.sty}{\usepackage{upquote}}{}
\IfFileExists{microtype.sty}{% use microtype if available
  \usepackage[]{microtype}
  \UseMicrotypeSet[protrusion]{basicmath} % disable protrusion for tt fonts
}{}
\makeatletter
\@ifundefined{KOMAClassName}{% if non-KOMA class
  \IfFileExists{parskip.sty}{%
    \usepackage{parskip}
  }{% else
    \setlength{\parindent}{0pt}
    \setlength{\parskip}{6pt plus 2pt minus 1pt}}
}{% if KOMA class
  \KOMAoptions{parskip=half}}
\makeatother
\usepackage{xcolor}
\usepackage[margin=1in]{geometry}
\usepackage{color}
\usepackage{fancyvrb}
\newcommand{\VerbBar}{|}
\newcommand{\VERB}{\Verb[commandchars=\\\{\}]}
\DefineVerbatimEnvironment{Highlighting}{Verbatim}{commandchars=\\\{\}}
% Add ',fontsize=\small' for more characters per line
\usepackage{framed}
\definecolor{shadecolor}{RGB}{248,248,248}
\newenvironment{Shaded}{\begin{snugshade}}{\end{snugshade}}
\newcommand{\AlertTok}[1]{\textcolor[rgb]{0.94,0.16,0.16}{#1}}
\newcommand{\AnnotationTok}[1]{\textcolor[rgb]{0.56,0.35,0.01}{\textbf{\textit{#1}}}}
\newcommand{\AttributeTok}[1]{\textcolor[rgb]{0.13,0.29,0.53}{#1}}
\newcommand{\BaseNTok}[1]{\textcolor[rgb]{0.00,0.00,0.81}{#1}}
\newcommand{\BuiltInTok}[1]{#1}
\newcommand{\CharTok}[1]{\textcolor[rgb]{0.31,0.60,0.02}{#1}}
\newcommand{\CommentTok}[1]{\textcolor[rgb]{0.56,0.35,0.01}{\textit{#1}}}
\newcommand{\CommentVarTok}[1]{\textcolor[rgb]{0.56,0.35,0.01}{\textbf{\textit{#1}}}}
\newcommand{\ConstantTok}[1]{\textcolor[rgb]{0.56,0.35,0.01}{#1}}
\newcommand{\ControlFlowTok}[1]{\textcolor[rgb]{0.13,0.29,0.53}{\textbf{#1}}}
\newcommand{\DataTypeTok}[1]{\textcolor[rgb]{0.13,0.29,0.53}{#1}}
\newcommand{\DecValTok}[1]{\textcolor[rgb]{0.00,0.00,0.81}{#1}}
\newcommand{\DocumentationTok}[1]{\textcolor[rgb]{0.56,0.35,0.01}{\textbf{\textit{#1}}}}
\newcommand{\ErrorTok}[1]{\textcolor[rgb]{0.64,0.00,0.00}{\textbf{#1}}}
\newcommand{\ExtensionTok}[1]{#1}
\newcommand{\FloatTok}[1]{\textcolor[rgb]{0.00,0.00,0.81}{#1}}
\newcommand{\FunctionTok}[1]{\textcolor[rgb]{0.13,0.29,0.53}{\textbf{#1}}}
\newcommand{\ImportTok}[1]{#1}
\newcommand{\InformationTok}[1]{\textcolor[rgb]{0.56,0.35,0.01}{\textbf{\textit{#1}}}}
\newcommand{\KeywordTok}[1]{\textcolor[rgb]{0.13,0.29,0.53}{\textbf{#1}}}
\newcommand{\NormalTok}[1]{#1}
\newcommand{\OperatorTok}[1]{\textcolor[rgb]{0.81,0.36,0.00}{\textbf{#1}}}
\newcommand{\OtherTok}[1]{\textcolor[rgb]{0.56,0.35,0.01}{#1}}
\newcommand{\PreprocessorTok}[1]{\textcolor[rgb]{0.56,0.35,0.01}{\textit{#1}}}
\newcommand{\RegionMarkerTok}[1]{#1}
\newcommand{\SpecialCharTok}[1]{\textcolor[rgb]{0.81,0.36,0.00}{\textbf{#1}}}
\newcommand{\SpecialStringTok}[1]{\textcolor[rgb]{0.31,0.60,0.02}{#1}}
\newcommand{\StringTok}[1]{\textcolor[rgb]{0.31,0.60,0.02}{#1}}
\newcommand{\VariableTok}[1]{\textcolor[rgb]{0.00,0.00,0.00}{#1}}
\newcommand{\VerbatimStringTok}[1]{\textcolor[rgb]{0.31,0.60,0.02}{#1}}
\newcommand{\WarningTok}[1]{\textcolor[rgb]{0.56,0.35,0.01}{\textbf{\textit{#1}}}}
\usepackage{longtable,booktabs,array}
\usepackage{calc} % for calculating minipage widths
% Correct order of tables after \paragraph or \subparagraph
\usepackage{etoolbox}
\makeatletter
\patchcmd\longtable{\par}{\if@noskipsec\mbox{}\fi\par}{}{}
\makeatother
% Allow footnotes in longtable head/foot
\IfFileExists{footnotehyper.sty}{\usepackage{footnotehyper}}{\usepackage{footnote}}
\makesavenoteenv{longtable}
\usepackage{graphicx}
\makeatletter
\def\maxwidth{\ifdim\Gin@nat@width>\linewidth\linewidth\else\Gin@nat@width\fi}
\def\maxheight{\ifdim\Gin@nat@height>\textheight\textheight\else\Gin@nat@height\fi}
\makeatother
% Scale images if necessary, so that they will not overflow the page
% margins by default, and it is still possible to overwrite the defaults
% using explicit options in \includegraphics[width, height, ...]{}
\setkeys{Gin}{width=\maxwidth,height=\maxheight,keepaspectratio}
% Set default figure placement to htbp
\makeatletter
\def\fps@figure{htbp}
\makeatother
\setlength{\emergencystretch}{3em} % prevent overfull lines
\providecommand{\tightlist}{%
  \setlength{\itemsep}{0pt}\setlength{\parskip}{0pt}}
\setcounter{secnumdepth}{5}
\ifLuaTeX
  \usepackage{selnolig}  % disable illegal ligatures
\fi
\IfFileExists{bookmark.sty}{\usepackage{bookmark}}{\usepackage{hyperref}}
\IfFileExists{xurl.sty}{\usepackage{xurl}}{} % add URL line breaks if available
\urlstyle{same}
\hypersetup{
  pdftitle={Problem Set 2: Omitted Variable Bias Key},
  pdfauthor={Claire Duquennois},
  hidelinks,
  pdfcreator={LaTeX via pandoc}}

\title{Problem Set 2: Omitted Variable Bias Key}
\author{Claire Duquennois}
\date{}

\begin{document}
\maketitle

\textbf{\emph{Group Member 1:}} Scott Heyman \textbf{\emph{Group Member
2:}} Xiang Li

\textbf{\emph{Group Member 3:}} Aileen Wang

\hypertarget{empirical-analysis-using-data-from-washington-2008-aer}{%
\section{Empirical Analysis using Data from Washington (2008,
AER)}\label{empirical-analysis-using-data-from-washington-2008-aer}}

This exercise uses data from Ebonya Washington's paper, ``Female
Socialization: How Daughters Affect their Legislator Father's voting on
Women's Issues,'' published in the \emph{American Economic Review} in
2008. This paper studies whether having a daughter affects legislator's
voting on women's issues.

\hypertarget{finding-the-data}{%
\section{Finding the data}\label{finding-the-data}}

The data can be found by following the link on the AER's website which
will take you to the ICPSR's data repository. You will need to sign in
to get access to the data files. Once logged in, you will find the set
of files that are typically included in a replication file. These
include several datasets, several .do files (which is a STATA command
file), and text files with the data descriptions which tell you about
the different variables included in the dataset. For this assignment we
will be using the \texttt{basic.dta} file.

Download it and save it in a `data' folder located in the same folder as
your project repository. Since some datasets in this course will be big,
we want to avoid keeping the data on github so I would recommend not
placing the data in the project repository itself.

\hypertarget{set-up-and-opening-the-data}{%
\section{Set up and opening the
data}\label{set-up-and-opening-the-data}}

Because this is a \texttt{.dta} file, you will need to open it with the
\texttt{read.dta} function that is included in the \texttt{haven}
packages.

Other packages you will need: \texttt{dplyr}, \texttt{stargazer} and
\texttt{lfe}.

Remember, if you have not used a package before you will need to install
the package as follows

\begin{Shaded}
\begin{Highlighting}[]
\CommentTok{\#install.packages(\textquotesingle{}haven\textquotesingle{},repos = "http://cran.us.r{-}project.org")}
\CommentTok{\#install.packages("dplyr",repos = "http://cran.us.r{-}project.org")}
\CommentTok{\#install.packages("stargazer",repos = "http://cran.us.r{-}project.org")}
\CommentTok{\#install.packages("lfe",repos = "http://cran.us.r{-}project.org")}
\end{Highlighting}
\end{Shaded}

Hint: Once you have run these once, on your machine, you may want to
comment them out with a \# so that your code runs faster.

This .Rmd file will be opened on different computers. But you don't want
to have to change the filepaths each time you pull a new version off of
GitHub. Because of this, I would recommend you avoid using any computer
specific filepaths in your code. Instead, make sure you and your
groupmates structure your project folders in the same way and only
specify filepaths within your project folder. R uses the folder where
you are saving your code as it's default ``working directory'' (where
things will be saved or be searched for unless specified otherwise). You
can move up to the parent folder by using \texttt{..} in the file path.
Thus, if your data is not saved in the forked github repository but is
saved in a folder called \texttt{data} next to it you can call your data
with the following file path: \texttt{"../data/basic.dta"}.

\hypertarget{question-now-that-the-packages-are-installed-call-all-your-packages-and-load-your-data.-how-many-observations-are-in-the-original-dataset}{%
\subsection{Question: Now that the packages are installed, call all your
packages and load your data. How many observations are in the original
dataset?}\label{question-now-that-the-packages-are-installed-call-all-your-packages-and-load-your-data.-how-many-observations-are-in-the-original-dataset}}

\textbf{Code:}

\begin{Shaded}
\begin{Highlighting}[]
\FunctionTok{library}\NormalTok{(}\StringTok{"haven"}\NormalTok{)}
\FunctionTok{library}\NormalTok{(}\StringTok{"dplyr"}\NormalTok{)}
\end{Highlighting}
\end{Shaded}

\begin{verbatim}
## 
## 载入程辑包:'dplyr'
\end{verbatim}

\begin{verbatim}
## The following objects are masked from 'package:stats':
## 
##     filter, lag
\end{verbatim}

\begin{verbatim}
## The following objects are masked from 'package:base':
## 
##     intersect, setdiff, setequal, union
\end{verbatim}

\begin{Shaded}
\begin{Highlighting}[]
\FunctionTok{library}\NormalTok{(}\StringTok{"stargazer"}\NormalTok{)}
\end{Highlighting}
\end{Shaded}

\begin{verbatim}
## 
## Please cite as:
\end{verbatim}

\begin{verbatim}
##  Hlavac, Marek (2022). stargazer: Well-Formatted Regression and Summary Statistics Tables.
\end{verbatim}

\begin{verbatim}
##  R package version 5.2.3. https://CRAN.R-project.org/package=stargazer
\end{verbatim}

\begin{Shaded}
\begin{Highlighting}[]
\FunctionTok{library}\NormalTok{(}\StringTok{"lfe"}\NormalTok{)}
\end{Highlighting}
\end{Shaded}

\begin{verbatim}
## 载入需要的程辑包:Matrix
\end{verbatim}

\begin{verbatim}
## Warning: 程辑包'Matrix'是用R版本4.3.2 来建造的
\end{verbatim}

\begin{Shaded}
\begin{Highlighting}[]
\NormalTok{basic }\OtherTok{\textless{}{-}} \FunctionTok{read\_dta}\NormalTok{(}\StringTok{"basic.dta"}\NormalTok{)}
\FunctionTok{count}\NormalTok{(basic)}
\end{Highlighting}
\end{Shaded}

\begin{verbatim}
## # A tibble: 1 x 1
##       n
##   <int>
## 1  1740
\end{verbatim}

\textbf{Answer:}\\
There are 1740 observations in the original dataset.

\clearpage

\hypertarget{cleaning-the-data}{%
\section{Cleaning the data}\label{cleaning-the-data}}

\hypertarget{question-the-original-dataset-contains-data-from-the-105th-to-108th-u.s.-congress.-we-only-use-the-observations-from-the-105th-congress.-refer-to-the-data-documentation-to-find-the-relevant-variable-and-then-use-the-filter-function-in-the-dplyr-package-to-extract-observations-from-the-105th-congress.}{%
\subsection{\texorpdfstring{Question: The original dataset contains data
from the 105th to 108th U.S. Congress. We only use the observations from
the 105th congress. Refer to the data documentation to find the relevant
variable and then use the \texttt{filter} function in the \texttt{dplyr}
package to extract observations from the 105th
congress.}{Question: The original dataset contains data from the 105th to 108th U.S. Congress. We only use the observations from the 105th congress. Refer to the data documentation to find the relevant variable and then use the filter function in the dplyr package to extract observations from the 105th congress.}}\label{question-the-original-dataset-contains-data-from-the-105th-to-108th-u.s.-congress.-we-only-use-the-observations-from-the-105th-congress.-refer-to-the-data-documentation-to-find-the-relevant-variable-and-then-use-the-filter-function-in-the-dplyr-package-to-extract-observations-from-the-105th-congress.}}

\textbf{Code:}

\begin{Shaded}
\begin{Highlighting}[]
\NormalTok{basic\_105 }\OtherTok{\textless{}{-}} \FunctionTok{filter}\NormalTok{(basic, congress }\SpecialCharTok{==} \StringTok{"105"}\NormalTok{)}
\FunctionTok{count}\NormalTok{(basic\_105)}
\end{Highlighting}
\end{Shaded}

\begin{verbatim}
## # A tibble: 1 x 1
##       n
##   <int>
## 1   435
\end{verbatim}

\clearpage

\hypertarget{questionthe-dataset-contains-many-variables-some-of-which-are-not-used-in-this-exercise.-keep-the-following-variables-in-the-final-dataset-hint-use-the-select-function-in-dplyr.}{%
\subsection{\texorpdfstring{Question:The dataset contains many
variables, some of which are not used in this exercise. Keep the
following variables in the final dataset (Hint: use the \texttt{select}
function in
\texttt{dplyr}).}{Question:The dataset contains many variables, some of which are not used in this exercise. Keep the following variables in the final dataset (Hint: use the select function in dplyr).}}\label{questionthe-dataset-contains-many-variables-some-of-which-are-not-used-in-this-exercise.-keep-the-following-variables-in-the-final-dataset-hint-use-the-select-function-in-dplyr.}}

\begin{longtable}[]{@{}
  >{\raggedright\arraybackslash}p{(\columnwidth - 2\tabcolsep) * \real{0.1220}}
  >{\raggedright\arraybackslash}p{(\columnwidth - 2\tabcolsep) * \real{0.8780}}@{}}
\toprule\noalign{}
\begin{minipage}[b]{\linewidth}\raggedright
Name
\end{minipage} & \begin{minipage}[b]{\linewidth}\raggedright
Description
\end{minipage} \\
\midrule\noalign{}
\endhead
\bottomrule\noalign{}
\endlastfoot
aauw & AAUW score \\
totchi & Total number of children \\
ngirls & Number of daughters \\
party & Political party. Democrats if 1, Republicans if 2, and
Independent if 3. \\
famale & Female dummy variable \\
white & White dummy variable \\
srvlng & Years of service \\
age & Age \\
demvote & State democratic vote share in most recent presidential
election \\
medinc & District median income \\
perf & Female proportion of district voting age population \\
perw & White proportion of total district population \\
perhs & High school graduate proportion of district population age 25 \\
percol & College graduate proportion of district population age 25 \\
perur & Urban proportion of total district population \\
moredef & State proportion who favor more defense spending \\
statabb & State abbreviation \\
district & id for electoral district \\
rgroup & religious group \\
region & region \\
\end{longtable}

\textbf{You can find the detailed description of each variable in the
original paper. The main variable in this analysis is \texttt{AAUW}, a
score created by the American Association of University Women (AAUW).
For each congress, AAUW selects pieces of legislation in the areas of
education, equality, and reproductive rights. The AAUW keeps track of
how each legislator voted on these pieces of legislation and whether
their vote aligned with the AAUW's position. The legislator's score is
equal to the proportion of these votes made in agreement with the AAUW.}

\textbf{Code:}

\begin{Shaded}
\begin{Highlighting}[]
\NormalTok{finaldata }\OtherTok{\textless{}{-}} \FunctionTok{select}\NormalTok{(basic\_105, aauw,totchi,ngirls,party,female,white,srvlng,age,demvote,medinc,perf,perw,perhs,percol,perur,moredef,statabb,district,rgroup,region)}
\end{Highlighting}
\end{Shaded}

\clearpage

\hypertarget{question-make-sure-your-final-dataset-is-a-data-frame.-you-can-check-your-datas-format-with-the-command-is.-if-the-first-element-of-the-returned-vector-is-not-data.frame-convert-your-dataset-with-the-function-as.data.frame.}{%
\subsection{\texorpdfstring{Question: Make sure your final dataset is a
data frame. You can check your data's format with the command
\texttt{is}. If the first element of the returned vector is not
``data.frame'', convert your dataset with the function
\texttt{as.data.frame}.}{Question: Make sure your final dataset is a data frame. You can check your data's format with the command is. If the first element of the returned vector is not ``data.frame'', convert your dataset with the function as.data.frame.}}\label{question-make-sure-your-final-dataset-is-a-data-frame.-you-can-check-your-datas-format-with-the-command-is.-if-the-first-element-of-the-returned-vector-is-not-data.frame-convert-your-dataset-with-the-function-as.data.frame.}}

\textbf{Code:}

\begin{Shaded}
\begin{Highlighting}[]
\FunctionTok{is}\NormalTok{(finaldata)}
\end{Highlighting}
\end{Shaded}

\begin{verbatim}
## [1] "tbl_df"     "tbl"        "data.frame" "list"       "oldClass"  
## [6] "vector"
\end{verbatim}

\clearpage

\hypertarget{summary-statistics}{%
\section{Summary Statistics}\label{summary-statistics}}

\hypertarget{question-report-summary-statistics-of-the-following-variables-in-the-dataset-political-party-age-race-gender-aauw-score-the-number-of-children-and-the-number-of-daughters.-present-these-summary-statistics-in-a-formatted-table-you-can-use-stargazer-or-other-packages.-make-this-table-as-communicative-as-possible.}{%
\subsection{\texorpdfstring{Question: Report summary statistics of the
following variables in the dataset: political party, age, race, gender,
AAUW score, the number of children, and the number of daughters. Present
these summary statistics in a formatted table, you can use
\texttt{stargazer} or other packages. Make this table as communicative
as
possible.}{Question: Report summary statistics of the following variables in the dataset: political party, age, race, gender, AAUW score, the number of children, and the number of daughters. Present these summary statistics in a formatted table, you can use stargazer or other packages. Make this table as communicative as possible.}}\label{question-report-summary-statistics-of-the-following-variables-in-the-dataset-political-party-age-race-gender-aauw-score-the-number-of-children-and-the-number-of-daughters.-present-these-summary-statistics-in-a-formatted-table-you-can-use-stargazer-or-other-packages.-make-this-table-as-communicative-as-possible.}}

Hints: If you want RMarkdown to display your outputted table, include
the code \texttt{results\ =\ "asis"} in the chunk header. This is true
for all chunks that output a formatted table. In the stargazer command,
you will want to specify the format of the table by including the code
\texttt{results="html"} for html output or \texttt{results="latex"} for
a pdf output.

\textbf{Code:}

\begin{Shaded}
\begin{Highlighting}[]
\FunctionTok{summary}\NormalTok{(finaldata, }\AttributeTok{results =} \StringTok{"asis"}\NormalTok{)}
\end{Highlighting}
\end{Shaded}

\begin{verbatim}
##       aauw            totchi           ngirls          party      
##  Min.   :  0.00   Min.   : 0.000   Min.   :0.000   Min.   :1.000  
##  1st Qu.:  0.00   1st Qu.: 2.000   1st Qu.:0.000   1st Qu.:1.000  
##  Median : 38.00   Median : 2.000   Median :1.000   Median :2.000  
##  Mean   : 47.31   Mean   : 2.493   Mean   :1.274   Mean   :1.529  
##  3rd Qu.:100.00   3rd Qu.: 3.000   3rd Qu.:2.000   3rd Qu.:2.000  
##  Max.   :100.00   Max.   :10.000   Max.   :7.000   Max.   :3.000  
##                   NA's   :1        NA's   :1                      
##      female           white           srvlng            age       
##  Min.   :0.0000   Min.   :0.000   Min.   : 1.000   Min.   :26.00  
##  1st Qu.:0.0000   1st Qu.:1.000   1st Qu.: 3.000   1st Qu.:45.00  
##  Median :0.0000   Median :1.000   Median : 5.000   Median :51.00  
##  Mean   :0.1103   Mean   :0.869   Mean   : 8.678   Mean   :51.67  
##  3rd Qu.:0.0000   3rd Qu.:1.000   3rd Qu.:13.000   3rd Qu.:58.00  
##  Max.   :1.0000   Max.   :1.000   Max.   :47.000   Max.   :87.00  
##                                                                   
##     demvote           medinc           perf             perw       
##  Min.   :0.2600   Min.   :16683   Min.   :0.4651   Min.   :0.1866  
##  1st Qu.:0.4200   1st Qu.:29593   1st Qu.:0.5129   1st Qu.:0.7563  
##  Median :0.4800   Median :34018   Median :0.5212   Median :0.8712  
##  Mean   :0.5032   Mean   :35923   Mean   :0.5208   Mean   :0.8043  
##  3rd Qu.:0.5700   3rd Qu.:40683   3rd Qu.:0.5297   3rd Qu.:0.9347  
##  Max.   :0.9400   Max.   :64199   Max.   :0.5677   Max.   :0.9871  
##                                                                    
##      perhs            percol           perur           moredef      
##  Min.   :0.3360   Min.   :0.0530   Min.   :0.1310   Min.   :0.0000  
##  1st Qu.:0.6985   1st Qu.:0.1435   1st Qu.:0.5650   1st Qu.:0.1716  
##  Median :0.7640   Median :0.1840   Median :0.7897   Median :0.2056  
##  Mean   :0.7501   Mean   :0.2006   Mean   :0.7515   Mean   :0.2141  
##  3rd Qu.:0.8125   3rd Qu.:0.2420   3rd Qu.:0.9768   3rd Qu.:0.2524  
##  Max.   :0.9230   Max.   :0.5140   Max.   :1.0000   Max.   :0.5052  
##                                                     NA's   :26      
##    statabb             district          rgroup         region         
##  Length:435         Min.   : 1.000   Min.   :0.000   Length:435        
##  Class :character   1st Qu.: 3.000   1st Qu.:1.000   Class :character  
##  Mode  :character   Median : 6.000   Median :1.000   Mode  :character  
##                     Mean   : 9.979   Mean   :1.529                     
##                     3rd Qu.:13.000   3rd Qu.:2.000                     
##                     Max.   :52.000   Max.   :4.000                     
## 
\end{verbatim}

\clearpage

\hypertarget{generate-variables}{%
\section{Generate Variables}\label{generate-variables}}

\hypertarget{questionconstruct-a-variable-called-repub_i-a-binary-set-to-1-if-the-observation-is-for-a-republican.}{%
\subsection{\texorpdfstring{Question:Construct a variable called
\(repub_i\), a binary set to 1 if the observation is for a
republican.}{Question:Construct a variable called repub\_i, a binary set to 1 if the observation is for a republican.}}\label{questionconstruct-a-variable-called-repub_i-a-binary-set-to-1-if-the-observation-is-for-a-republican.}}

\textbf{Code:}

\begin{Shaded}
\begin{Highlighting}[]
\NormalTok{finaldata}\SpecialCharTok{$}\NormalTok{repub\_i }\OtherTok{\textless{}{-}} \FunctionTok{ifelse}\NormalTok{(finaldata}\SpecialCharTok{$}\NormalTok{party }\SpecialCharTok{==} \DecValTok{2}\NormalTok{, }\DecValTok{1}\NormalTok{, }\DecValTok{0}\NormalTok{)}
\end{Highlighting}
\end{Shaded}

\clearpage

\hypertarget{run-estimations}{%
\section{Run Estimations}\label{run-estimations}}

\hypertarget{question-2-pages-estimate-the-following-linear-regression-models-using-the-felm-command-part-of-the-lfe-package.-report-your-regression-results-in-a-formatted-table-using-a-package-such-as-stargazer.-report-robust-standard-errors-in-your-table-hint-in-stargazer-specify-se-listmodel1rse-model2rse-model3rse.-make-this-table-as-informative-as-possible-by-adding-needed-information-and-removing-superfluous-information.}{%
\subsection{\texorpdfstring{Question: (2 pages) Estimate the following
linear regression models using the \texttt{felm} command (part of the
lfe package). Report your regression results in a formatted table using
a package such as \texttt{stargazer}. Report robust standard errors in
your table (Hint: in stargazer specify
\texttt{se\ =\ list(model1\$rse,\ model2\$rse,\ model3\$rse)}). Make
this table as informative as possible by adding needed information and
removing superfluous
information.}{Question: (2 pages) Estimate the following linear regression models using the felm command (part of the lfe package). Report your regression results in a formatted table using a package such as stargazer. Report robust standard errors in your table (Hint: in stargazer specify se = list(model1\$rse, model2\$rse, model3\$rse)). Make this table as informative as possible by adding needed information and removing superfluous information.}}\label{question-2-pages-estimate-the-following-linear-regression-models-using-the-felm-command-part-of-the-lfe-package.-report-your-regression-results-in-a-formatted-table-using-a-package-such-as-stargazer.-report-robust-standard-errors-in-your-table-hint-in-stargazer-specify-se-listmodel1rse-model2rse-model3rse.-make-this-table-as-informative-as-possible-by-adding-needed-information-and-removing-superfluous-information.}}

\[
\begin{aligned}
 aauw_i&=\beta_0+\beta_1ngirls_i+\epsilon_i\\
 aauw_i&=\beta_0+\beta_1ngirls_i+\beta_2totchi+\epsilon_i\\
  aauw_i&=\beta_0+\beta_1ngirls_i+\beta_2totchi+\beta_3female_i+\beta_4repub_i+\epsilon_i\\
\end{aligned}
\]

\textbf{Code:}

\begin{Shaded}
\begin{Highlighting}[]
\NormalTok{reg}\FloatTok{.1} \OtherTok{\textless{}{-}} \FunctionTok{felm}\NormalTok{(aauw }\SpecialCharTok{\textasciitilde{}}\NormalTok{ ngirls, finaldata)}
\NormalTok{reg}\FloatTok{.2} \OtherTok{\textless{}{-}} \FunctionTok{felm}\NormalTok{(aauw }\SpecialCharTok{\textasciitilde{}}\NormalTok{ ngirls}\SpecialCharTok{+}\NormalTok{totchi, finaldata)}
\NormalTok{reg}\FloatTok{.3} \OtherTok{\textless{}{-}} \FunctionTok{felm}\NormalTok{(aauw }\SpecialCharTok{\textasciitilde{}}\NormalTok{ ngirls}\SpecialCharTok{+}\NormalTok{totchi}\SpecialCharTok{+}\NormalTok{female}\SpecialCharTok{+}\NormalTok{repub\_i, finaldata)}

\FunctionTok{stargazer}\NormalTok{(reg}\FloatTok{.1}\NormalTok{, reg}\FloatTok{.2}\NormalTok{, reg}\FloatTok{.3}\NormalTok{, }\AttributeTok{header =} \ConstantTok{FALSE}\NormalTok{, }\AttributeTok{type =} \StringTok{"text"}\NormalTok{, }\AttributeTok{se =} \FunctionTok{list}\NormalTok{(reg}\FloatTok{.1}\SpecialCharTok{$}\NormalTok{rse, reg}\FloatTok{.2}\SpecialCharTok{$}\NormalTok{rse, reg}\FloatTok{.3}\SpecialCharTok{$}\NormalTok{rse))}
\end{Highlighting}
\end{Shaded}

\begin{verbatim}
## 
## =========================================================================
##                                      Dependent variable:                 
##                     -----------------------------------------------------
##                                             aauw                         
##                            (1)               (2)               (3)       
## -------------------------------------------------------------------------
## ngirls                   -2.784            5.776**           2.825**     
##                          (1.750)           (2.714)           (1.306)     
##                                                                          
## totchi                                    -7.992***         -3.149***    
##                                            (1.784)           (0.964)     
##                                                                          
## female                                                      12.577***    
##                                                              (3.258)     
##                                                                          
## repub_i                                                    -71.783***    
##                                                              (2.100)     
##                                                                          
## Constant                50.964***         59.982***         87.822***    
##                          (3.036)           (3.520)           (1.809)     
##                                                                          
## -------------------------------------------------------------------------
## Observations               434               434               434       
## R2                        0.006             0.051             0.796      
## Adjusted R2               0.003             0.047             0.794      
## Residual Std. Error 41.939 (df = 432) 41.010 (df = 431) 19.055 (df = 429)
## =========================================================================
## Note:                                         *p<0.1; **p<0.05; ***p<0.01
\end{verbatim}

\begin{Shaded}
\begin{Highlighting}[]
\NormalTok{reg}\FloatTok{.1} \OtherTok{\textless{}{-}} \FunctionTok{felm}\NormalTok{(aauw }\SpecialCharTok{\textasciitilde{}}\NormalTok{ ngirls, finaldata)}
\NormalTok{reg}\FloatTok{.2} \OtherTok{\textless{}{-}} \FunctionTok{felm}\NormalTok{(aauw }\SpecialCharTok{\textasciitilde{}}\NormalTok{ ngirls}\SpecialCharTok{|}\NormalTok{totchi, finaldata)}
\NormalTok{reg}\FloatTok{.3} \OtherTok{\textless{}{-}} \FunctionTok{felm}\NormalTok{(aauw }\SpecialCharTok{\textasciitilde{}}\NormalTok{ ngirls}\SpecialCharTok{|}\NormalTok{totchi}\SpecialCharTok{+}\NormalTok{female}\SpecialCharTok{+}\NormalTok{repub\_i, finaldata)}

\FunctionTok{stargazer}\NormalTok{(reg}\FloatTok{.1}\NormalTok{, reg}\FloatTok{.2}\NormalTok{, reg}\FloatTok{.3}\NormalTok{, }\AttributeTok{header =} \ConstantTok{FALSE}\NormalTok{, }\AttributeTok{type =} \StringTok{"text"}\NormalTok{, }\AttributeTok{se =} \FunctionTok{list}\NormalTok{(reg}\FloatTok{.1}\SpecialCharTok{$}\NormalTok{rse, reg}\FloatTok{.2}\SpecialCharTok{$}\NormalTok{rse, reg}\FloatTok{.3}\SpecialCharTok{$}\NormalTok{rse))}
\end{Highlighting}
\end{Shaded}

\begin{verbatim}
## 
## =========================================================================
##                                      Dependent variable:                 
##                     -----------------------------------------------------
##                                             aauw                         
##                            (1)               (2)               (3)       
## -------------------------------------------------------------------------
## ngirls                   -2.784            5.748**           3.043**     
##                          (1.750)           (2.667)           (1.359)     
##                                                                          
## Constant                50.964***                                        
##                          (3.036)                                         
##                                                                          
## -------------------------------------------------------------------------
## Observations               434               434               434       
## R2                        0.006             0.065             0.799      
## Adjusted R2               0.003             0.040             0.793      
## Residual Std. Error 41.939 (df = 432) 41.154 (df = 422) 19.122 (df = 420)
## =========================================================================
## Note:                                         *p<0.1; **p<0.05; ***p<0.01
\end{verbatim}

\clearpage

\hypertarget{question-2-pages-compare-the-ols-estimates-of-beta_1-across-the-above-three-specifications.-discuss-what-explains-the-difference-if-any-of-the-estimate-across-three-specifications-which-control-variable-is-particularly-important-and-why}{%
\subsection{\texorpdfstring{Question: (2 pages) Compare the OLS
estimates of \(\beta_1\) across the above three specifications. Discuss
what explains the difference (if any) of the estimate across three
specifications? Which control variable is particularly important and
why?}{Question: (2 pages) Compare the OLS estimates of \textbackslash beta\_1 across the above three specifications. Discuss what explains the difference (if any) of the estimate across three specifications? Which control variable is particularly important and why?}}\label{question-2-pages-compare-the-ols-estimates-of-beta_1-across-the-above-three-specifications.-discuss-what-explains-the-difference-if-any-of-the-estimate-across-three-specifications-which-control-variable-is-particularly-important-and-why}}

\textbf{Code:}

\begin{Shaded}
\begin{Highlighting}[]
\NormalTok{beta1\_model1 }\OtherTok{\textless{}{-}} \FunctionTok{coef}\NormalTok{(reg}\FloatTok{.1}\NormalTok{)[}\StringTok{"ngirls"}\NormalTok{]}
\NormalTok{beta1\_model2 }\OtherTok{\textless{}{-}} \FunctionTok{coef}\NormalTok{(reg}\FloatTok{.2}\NormalTok{)[}\StringTok{"ngirls"}\NormalTok{]}
\NormalTok{beta1\_model3 }\OtherTok{\textless{}{-}} \FunctionTok{coef}\NormalTok{(reg}\FloatTok{.3}\NormalTok{)[}\StringTok{"ngirls"}\NormalTok{]}

\FunctionTok{c}\NormalTok{(beta1\_model1, beta1\_model2, beta1\_model3)}
\end{Highlighting}
\end{Shaded}

\begin{verbatim}
##    ngirls    ngirls    ngirls 
## -2.783935  5.748309  3.042570
\end{verbatim}

\textbf{Answer:} Control variables are different from the three models,
the first model only considered ngirls (number of daughters), the second
one considered ngirls and totchi (total number of children), the third
one also included these two variables and added female and repub\_i(two
dummy variables). If we just look into the first model, it may seem that
the number of girls has negative correlation, however, once we run the
other models, it becomes clear that there was likely omitted variable
bias and that the number of girls is positvely correlated.

repub\_i is important because it has the largest influence and it's
significant.

\clearpage

\hypertarget{question-consider-the-third-specification-with-3-controls-in-addition-to-ngirls_i.-conditional-on-the-number-of-children-and-other-variables-do-you-think-ngirls_i-is-plausibly-exogenous-what-is-the-identifying-assumption-necessary-for-beta_1-to-be-interpreted-as-a-causal-estimate-what-evidence-does-washington-give-to-support-this-assumption}{%
\subsection{\texorpdfstring{Question: Consider the third specification
(with 3 controls in addition to \(ngirls_i\). Conditional on the number
of children and other variables, do you think \(ngirls_i\) is plausibly
exogenous? What is the identifying assumption necessary for \(\beta_1\)
to be interpreted as a causal estimate? What evidence does Washington
give to support this
assumption?}{Question: Consider the third specification (with 3 controls in addition to ngirls\_i. Conditional on the number of children and other variables, do you think ngirls\_i is plausibly exogenous? What is the identifying assumption necessary for \textbackslash beta\_1 to be interpreted as a causal estimate? What evidence does Washington give to support this assumption?}}\label{question-consider-the-third-specification-with-3-controls-in-addition-to-ngirls_i.-conditional-on-the-number-of-children-and-other-variables-do-you-think-ngirls_i-is-plausibly-exogenous-what-is-the-identifying-assumption-necessary-for-beta_1-to-be-interpreted-as-a-causal-estimate-what-evidence-does-washington-give-to-support-this-assumption}}

\textbf{Answer:} \(ngirls_i\) is plausibly exogenous. Assumption:
conditional on number of children, the number of female children is a
random variable. According to the data in the appendix, representatives
did not follow the rule of stopping childbearing based on preference for
children, and voters did not consider the gender mix of children in
their selection of representatives.

\clearpage

\hypertarget{question-2-pages-it-is-possible-that-the-effects-of-having-daughters-might-be-different-for-female-and-male-legislators.-estimate-four-different-models-to-think-about-this-question-the-equivalent-of-model-3-separately-on-men-and-women-model-3-with-a-single-interaction-term-added-and-model-3-with-three-interaction-terms-added.-present-your-results-in-a-table.-is-there-evidence-that-the-effect-of-a-daughter-differs-for-male-and-female-legislators-of-the-four-models-you-estimated-which-are-equivalent-which-are-different-and-why}{%
\subsection{Question: (2 pages) It is possible that the effects of
having daughters might be different for female and male legislators.
Estimate four different models to think about this question: the
equivalent of model 3 separately on men and women, model 3 with a single
interaction term added, and model 3 with three interaction terms added.
Present your results in a table. Is there evidence that the effect of a
daughter differs for male and female legislators? Of the four models you
estimated, which are equivalent, which are different, and
why?}\label{question-2-pages-it-is-possible-that-the-effects-of-having-daughters-might-be-different-for-female-and-male-legislators.-estimate-four-different-models-to-think-about-this-question-the-equivalent-of-model-3-separately-on-men-and-women-model-3-with-a-single-interaction-term-added-and-model-3-with-three-interaction-terms-added.-present-your-results-in-a-table.-is-there-evidence-that-the-effect-of-a-daughter-differs-for-male-and-female-legislators-of-the-four-models-you-estimated-which-are-equivalent-which-are-different-and-why}}

\textbf{Code:}

\begin{Shaded}
\begin{Highlighting}[]
\NormalTok{data\_male }\OtherTok{\textless{}{-}} \FunctionTok{filter}\NormalTok{(finaldata, female }\SpecialCharTok{==} \DecValTok{0}\NormalTok{)}
\NormalTok{data\_female }\OtherTok{\textless{}{-}} \FunctionTok{filter}\NormalTok{(finaldata, female }\SpecialCharTok{==} \DecValTok{1}\NormalTok{)}

\NormalTok{reg.m }\OtherTok{\textless{}{-}} \FunctionTok{felm}\NormalTok{(aauw }\SpecialCharTok{\textasciitilde{}}\NormalTok{ ngirls}\SpecialCharTok{+}\NormalTok{totchi}\SpecialCharTok{+}\NormalTok{female}\SpecialCharTok{+}\NormalTok{repub\_i, data\_male)}
\end{Highlighting}
\end{Shaded}

\begin{verbatim}
## Warning in chol.default(mat, pivot = TRUE, tol = tol): the matrix is either
## rank-deficient or not positive definite
\end{verbatim}

\begin{Shaded}
\begin{Highlighting}[]
\NormalTok{reg.f }\OtherTok{\textless{}{-}} \FunctionTok{felm}\NormalTok{(aauw }\SpecialCharTok{\textasciitilde{}}\NormalTok{ ngirls}\SpecialCharTok{+}\NormalTok{totchi}\SpecialCharTok{+}\NormalTok{female}\SpecialCharTok{+}\NormalTok{repub\_i, data\_female)}
\end{Highlighting}
\end{Shaded}

\begin{verbatim}
## Warning in chol.default(mat, pivot = TRUE, tol = tol): the matrix is either
## rank-deficient or not positive definite
\end{verbatim}

\begin{Shaded}
\begin{Highlighting}[]
\NormalTok{reg.mn }\OtherTok{\textless{}{-}} \FunctionTok{felm}\NormalTok{(aauw }\SpecialCharTok{\textasciitilde{}}\NormalTok{ ngirls}\SpecialCharTok{+}\NormalTok{totchi}\SpecialCharTok{+}\NormalTok{female}\SpecialCharTok{+}\NormalTok{repub\_i}\SpecialCharTok{+}\NormalTok{female}\SpecialCharTok{*}\NormalTok{ngirls, finaldata)}
\NormalTok{reg}\FloatTok{.3}\NormalTok{n }\OtherTok{\textless{}{-}} \FunctionTok{felm}\NormalTok{(aauw }\SpecialCharTok{\textasciitilde{}}\NormalTok{ ngirls}\SpecialCharTok{+}\NormalTok{totchi}\SpecialCharTok{+}\NormalTok{female}\SpecialCharTok{+}\NormalTok{repub\_i}\SpecialCharTok{+}\NormalTok{female}\SpecialCharTok{*}\NormalTok{ngirls}\SpecialCharTok{+}\NormalTok{totchi}\SpecialCharTok{*}\NormalTok{ngirls}\SpecialCharTok{+}\NormalTok{repub\_i}\SpecialCharTok{*}\NormalTok{ngirls, finaldata)}

\FunctionTok{stargazer}\NormalTok{(reg.m, reg.f, reg.mn,reg}\FloatTok{.3}\NormalTok{n, }\AttributeTok{header =} \ConstantTok{FALSE}\NormalTok{, }\AttributeTok{type =} \StringTok{"text"}\NormalTok{, }\AttributeTok{se =} \FunctionTok{list}\NormalTok{(reg.m}\SpecialCharTok{$}\NormalTok{rse, reg.f}\SpecialCharTok{$}\NormalTok{rse,reg.mn}\SpecialCharTok{$}\NormalTok{rse,reg}\FloatTok{.3}\NormalTok{n}\SpecialCharTok{$}\NormalTok{rse))}
\end{Highlighting}
\end{Shaded}

\begin{verbatim}
## 
## ==========================================================================================
##                                              Dependent variable:                          
##                     ----------------------------------------------------------------------
##                                                      aauw                                 
##                            (1)              (2)               (3)               (4)       
## ------------------------------------------------------------------------------------------
## ngirls                   3.021**           -0.246           2.706**           4.142**     
##                          (1.302)          (6.281)           (1.281)           (1.721)     
##                                                                                           
## totchi                  -3.417***          1.222           -3.133***         -2.328**     
##                          (0.986)          (3.971)           (0.958)           (1.042)     
##                                                                                           
## female                                                     11.103**          10.895**     
##                          (0.000)          (0.000)           (4.659)           (4.796)     
##                                                                                           
## repub_i                -71.943***        -70.881***       -71.791***        -74.316***    
##                          (2.102)          (9.462)           (2.098)           (3.021)     
##                                                                                           
## ngirls:female                                                1.097             1.036      
##                                                             (3.295)           (3.463)     
##                                                                                           
## ngirls:totchi                                                                 -0.704*     
##                                                                               (0.363)     
##                                                                                           
## ngirls:repub_i                                                                 2.166      
##                                                                               (1.971)     
##                                                                                           
## Constant                88.337***        94.006***         87.936***         87.111***    
##                          (1.810)          (3.441)           (1.831)           (2.362)     
##                                                                                           
## ------------------------------------------------------------------------------------------
## Observations               386               48               434               434       
## R2                        0.791            0.729             0.796             0.798      
## Adjusted R2               0.789            0.711             0.794             0.795      
## Residual Std. Error 18.920 (df = 382) 20.365 (df = 44) 19.074 (df = 428) 19.020 (df = 426)
## ==========================================================================================
## Note:                                                          *p<0.1; **p<0.05; ***p<0.01
\end{verbatim}

\textbf{Answer:} It seems as if there is a difference between male and
females with males being effected more by having daughters than females.
3 is the same as 1 and 2 if applied to the male and female dataset
respectivly.

\clearpage

\hypertarget{fixed-effects}{%
\section{Fixed Effects:}\label{fixed-effects}}

\hypertarget{question-2-pages-equation-1-from-washingtons-paper-is-a-little-bit-different-from-the-equations-you-have-estimated-so-far.-estimate-the-three-models-specified-below-where-gamma_i-is-a-fixed-effect-for-the-number-of-children.-present-your-results-in-a-table-and-explain-the-difference-between-the-three-models.}{%
\subsection{\texorpdfstring{Question: (2 pages) Equation 1 from
Washington's paper is a little bit different from the equations you have
estimated so far. Estimate the three models specified below (where
\(\gamma_i\) is a fixed effect for the number of children). Present your
results in a table and explain the difference between the three
models.}{Question: (2 pages) Equation 1 from Washington's paper is a little bit different from the equations you have estimated so far. Estimate the three models specified below (where \textbackslash gamma\_i is a fixed effect for the number of children). Present your results in a table and explain the difference between the three models.}}\label{question-2-pages-equation-1-from-washingtons-paper-is-a-little-bit-different-from-the-equations-you-have-estimated-so-far.-estimate-the-three-models-specified-below-where-gamma_i-is-a-fixed-effect-for-the-number-of-children.-present-your-results-in-a-table-and-explain-the-difference-between-the-three-models.}}

\[
\begin{aligned}
 aauw_i&=\beta_0+\beta_1ngirls_i+\beta_2totchi+\epsilon_i\\
  aauw_i&=\beta_0+\beta_1ngirls_i+\beta_2chi1+...+\beta_{10}chi10 +\epsilon_i\\
    aauw_i&=\beta_0+\beta_1ngirls_i+\gamma_i+\epsilon_i\\
\end{aligned}
\]

Hint: you will need to generate the dummy variables for the second
equation or code it as \texttt{factor()}. For the third equation, the
\texttt{felm} function allows you to specify fixed effects.

\textbf{Code:}

\begin{Shaded}
\begin{Highlighting}[]
\NormalTok{reg}\FloatTok{.1} \OtherTok{\textless{}{-}} \FunctionTok{felm}\NormalTok{(aauw }\SpecialCharTok{\textasciitilde{}}\NormalTok{ ngirls}\SpecialCharTok{+}\NormalTok{totchi, finaldata)}
\NormalTok{reg}\FloatTok{.2} \OtherTok{\textless{}{-}} \FunctionTok{felm}\NormalTok{(aauw }\SpecialCharTok{\textasciitilde{}}\NormalTok{ ngirls}\SpecialCharTok{+}\FunctionTok{as.factor}\NormalTok{(totchi), finaldata)}
\NormalTok{reg}\FloatTok{.3} \OtherTok{\textless{}{-}} \FunctionTok{felm}\NormalTok{(aauw }\SpecialCharTok{\textasciitilde{}}\NormalTok{ ngirls}\SpecialCharTok{|}\NormalTok{totchi, finaldata)}

\FunctionTok{stargazer}\NormalTok{(reg}\FloatTok{.1}\NormalTok{, reg}\FloatTok{.2}\NormalTok{, reg}\FloatTok{.3}\NormalTok{, }\AttributeTok{header =} \ConstantTok{FALSE}\NormalTok{, }\AttributeTok{type =} \StringTok{"text"}\NormalTok{, }\AttributeTok{se =} \FunctionTok{list}\NormalTok{(reg}\FloatTok{.1}\SpecialCharTok{$}\NormalTok{rse, reg}\FloatTok{.2}\SpecialCharTok{$}\NormalTok{rse, reg}\FloatTok{.3}\SpecialCharTok{$}\NormalTok{rse))}
\end{Highlighting}
\end{Shaded}

\begin{verbatim}
## 
## =========================================================================
##                                      Dependent variable:                 
##                     -----------------------------------------------------
##                                             aauw                         
##                            (1)               (2)               (3)       
## -------------------------------------------------------------------------
## ngirls                   5.776**           5.748**           5.748**     
##                          (2.714)           (2.667)           (2.667)     
##                                                                          
## totchi                  -7.992***                                        
##                          (1.784)                                         
##                                                                          
## as.factor(totchi)1                          7.616                        
##                                            (8.816)                       
##                                                                          
## as.factor(totchi)2                         -6.182                        
##                                            (7.074)                       
##                                                                          
## as.factor(totchi)3                        -17.186**                      
##                                            (7.770)                       
##                                                                          
## as.factor(totchi)4                       -25.833***                      
##                                            (9.090)                       
##                                                                          
## as.factor(totchi)5                        -28.128**                      
##                                           (11.601)                       
##                                                                          
## as.factor(totchi)6                         -34.712                       
##                                           (24.334)                       
##                                                                          
## as.factor(totchi)7                       -65.986***                      
##                                           (11.828)                       
##                                                                          
## as.factor(totchi)8                       -74.859***                      
##                                           (15.283)                       
##                                                                          
## as.factor(totchi)9                       -81.108***                      
##                                           (14.386)                       
##                                                                          
## as.factor(totchi)10                      -75.360***                      
##                                           (11.957)                       
##                                                                          
## Constant                59.982***         52.367***                      
##                          (3.520)           (5.400)                       
##                                                                          
## -------------------------------------------------------------------------
## Observations               434               434               434       
## R2                        0.051             0.065             0.065      
## Adjusted R2               0.047             0.040             0.040      
## Residual Std. Error 41.010 (df = 431) 41.154 (df = 422) 41.154 (df = 422)
## =========================================================================
## Note:                                         *p<0.1; **p<0.05; ***p<0.01
\end{verbatim}

\textbf{Answer:} The first model looks at the number of girls and the
total number of childern with no dummy variables. The seccond model has
a dummy variable for each total amount of kids you could have (1-10).
The third looks at the number of kids as a fixed effect.

\clearpage

\hypertarget{question-2-pages-reproduce-the-results-in-column-2-of-table-2-from-washingtons-paper.}{%
\subsection{Question: (2 pages) Reproduce the results in column 2 of
table 2 from Washington's
paper.}\label{question-2-pages-reproduce-the-results-in-column-2-of-table-2-from-washingtons-paper.}}

\textbf{Code:}

\begin{Shaded}
\begin{Highlighting}[]
\CommentTok{\# add age square and srvlng to data }
\NormalTok{finaldata}\SpecialCharTok{$}\NormalTok{age2 }\OtherTok{\textless{}{-}}\NormalTok{ finaldata}\SpecialCharTok{$}\NormalTok{age}\SpecialCharTok{\^{}}\DecValTok{2}
\NormalTok{finaldata}\SpecialCharTok{$}\NormalTok{srvlng2 }\OtherTok{\textless{}{-}}\NormalTok{ finaldata}\SpecialCharTok{$}\NormalTok{srvlng}\SpecialCharTok{\^{}}\DecValTok{2}

\CommentTok{\#reg.2 \textless{}{-} felm(aauw \textasciitilde{} ngirls+as.factor(totchi), finaldata)}

\NormalTok{reg.t2 }\OtherTok{\textless{}{-}} \FunctionTok{felm}\NormalTok{(}\AttributeTok{data =}\NormalTok{ finaldata, aauw }\SpecialCharTok{\textasciitilde{}}\NormalTok{ ngirls}\SpecialCharTok{+}\NormalTok{female}\SpecialCharTok{+}\NormalTok{white}\SpecialCharTok{+}\NormalTok{repub\_i}\SpecialCharTok{+}\NormalTok{age}\SpecialCharTok{+}\NormalTok{age2}\SpecialCharTok{+}\NormalTok{srvlng}\SpecialCharTok{+}\NormalTok{srvlng2}\SpecialCharTok{+}\FunctionTok{as.factor}\NormalTok{(rgroup)}\SpecialCharTok{+}\NormalTok{demvote}\SpecialCharTok{|}\NormalTok{totchi}\SpecialCharTok{+}\NormalTok{region)}

\FunctionTok{stargazer}\NormalTok{(reg.t2, }\AttributeTok{header =} \ConstantTok{FALSE}\NormalTok{, }\AttributeTok{type =} \StringTok{"text"}\NormalTok{, }\AttributeTok{se =} \FunctionTok{list}\NormalTok{(reg.t2}\SpecialCharTok{$}\NormalTok{rse))}
\end{Highlighting}
\end{Shaded}

\begin{verbatim}
## 
## ===============================================
##                         Dependent variable:    
##                     ---------------------------
##                                aauw            
## -----------------------------------------------
## ngirls                        2.385**          
##                               (1.198)          
##                                                
## female                       9.194***          
##                               (3.336)          
##                                                
## white                          0.144           
##                               (3.539)          
##                                                
## repub_i                     -60.468***         
##                               (2.993)          
##                                                
## age                            0.854           
##                               (0.997)          
##                                                
## age2                          -0.006           
##                               (0.010)          
##                                                
## srvlng                        -0.208           
##                               (0.320)          
##                                                
## srvlng2                        0.004           
##                               (0.012)          
##                                                
## as.factor(rgroup)1            -5.671           
##                               (3.468)          
##                                                
## as.factor(rgroup)2          -10.175***         
##                               (3.312)          
##                                                
## as.factor(rgroup)3            -2.466           
##                               (5.740)          
##                                                
## as.factor(rgroup)4             4.012           
##                               (3.926)          
##                                                
## demvote                      62.148***         
##                              (13.065)          
##                                                
## -----------------------------------------------
## Observations                    434            
## R2                             0.840           
## Adjusted R2                    0.828           
## Residual Std. Error      17.441 (df = 402)     
## ===============================================
## Note:               *p<0.1; **p<0.05; ***p<0.01
\end{verbatim}

\clearpage

\hypertarget{question-explain-what-the-region-fixed-effects-are-controlling-for}{%
\subsection{Question: Explain what the region fixed effects are
controlling
for?}\label{question-explain-what-the-region-fixed-effects-are-controlling-for}}

\textbf{Answer:} They are controlling for the differences in regions.
Rather than comparing all of the regions it is comparing the different
congress members within their respective regions.

\clearpage

\hypertarget{question-2-pages-reload-the-data-and-this-time-we-will-keep-observations-from-all-of-the-congresses.-generate-a-variable-that-creates-a-unique-identifier-for-region-by-year.-estimate-the-following-models-and-present-your-results-in-a-table.}{%
\subsection{Question: (2 pages) Reload the data and this time we will
keep observations from all of the congresses. Generate a variable that
creates a unique identifier for region by year. Estimate the following
models and present your results in a
table.}\label{question-2-pages-reload-the-data-and-this-time-we-will-keep-observations-from-all-of-the-congresses.-generate-a-variable-that-creates-a-unique-identifier-for-region-by-year.-estimate-the-following-models-and-present-your-results-in-a-table.}}

\[
\begin{aligned}
    aauw_i&=\beta_0+\beta_1ngirls_i+\gamma_i+\phi_i+\epsilon_i\\
    aauw_i&=\beta_0+\beta_1ngirls_i+\gamma_i+\phi_i+\eta_i+\epsilon_i\\
    aauw_i&=\beta_0+\beta_1ngirls_i+\gamma_i+\theta_i+\epsilon_i\\
    aauw_i&=\beta_0+\beta_1ngirls_i+\rho_i+\epsilon_i\\
\end{aligned}
\]

\textbf{\(\gamma_i\) is a fixed effect for the total number of children,
\(\phi_i\) is a region fixed effect, \(\eta_i\) is a year (congress
session) fixed effect and \(\theta_i\) is a region by year fixed effect
and \(\rho_i\) is a total children by region by year fixed effect.
Explain what the differences between these four different estimation. Is
there a downside to a specification like the fourth specification? }

\textbf{Code:}

\begin{Shaded}
\begin{Highlighting}[]
\CommentTok{\# build year fixed effect}
\NormalTok{data8 }\OtherTok{\textless{}{-}}\NormalTok{ basic}

\NormalTok{data8}\SpecialCharTok{$}\NormalTok{uniqueYearRegion }\OtherTok{\textless{}{-}} \FunctionTok{paste}\NormalTok{(data8}\SpecialCharTok{$}\NormalTok{year, data8}\SpecialCharTok{$}\NormalTok{region, }\AttributeTok{sep=}\StringTok{"\_"}\NormalTok{)}
\NormalTok{data8}\SpecialCharTok{$}\NormalTok{uniqueYearRegionChildren }\OtherTok{\textless{}{-}} \FunctionTok{paste}\NormalTok{(data8}\SpecialCharTok{$}\NormalTok{year, data8}\SpecialCharTok{$}\NormalTok{region, data8}\SpecialCharTok{$}\NormalTok{totchi, }\AttributeTok{sep=}\StringTok{"\_"}\NormalTok{)}


\CommentTok{\# make regressions }
\NormalTok{reg}\FloatTok{.1} \OtherTok{\textless{}{-}} \FunctionTok{felm}\NormalTok{(}\AttributeTok{data =}\NormalTok{ data8, aauw }\SpecialCharTok{\textasciitilde{}}\NormalTok{ ngirls}\SpecialCharTok{|}\NormalTok{totchi}\SpecialCharTok{+}\NormalTok{region)}
\NormalTok{reg}\FloatTok{.2} \OtherTok{\textless{}{-}} \FunctionTok{felm}\NormalTok{(}\AttributeTok{data =}\NormalTok{ data8, aauw }\SpecialCharTok{\textasciitilde{}}\NormalTok{ ngirls}\SpecialCharTok{|}\NormalTok{totchi}\SpecialCharTok{+}\NormalTok{region}\SpecialCharTok{+}\NormalTok{congress)}
\NormalTok{reg}\FloatTok{.3} \OtherTok{\textless{}{-}} \FunctionTok{felm}\NormalTok{(}\AttributeTok{data =}\NormalTok{ data8, aauw }\SpecialCharTok{\textasciitilde{}}\NormalTok{ ngirls}\SpecialCharTok{|}\NormalTok{totchi}\SpecialCharTok{+}\NormalTok{region}\SpecialCharTok{+}\NormalTok{uniqueYearRegion)}
\NormalTok{reg}\FloatTok{.4} \OtherTok{\textless{}{-}} \FunctionTok{felm}\NormalTok{(}\AttributeTok{data =}\NormalTok{ data8, aauw }\SpecialCharTok{\textasciitilde{}}\NormalTok{ ngirls}\SpecialCharTok{|}\NormalTok{uniqueYearRegionChildren)}

\FunctionTok{stargazer}\NormalTok{(reg}\FloatTok{.1}\NormalTok{, reg}\FloatTok{.2}\NormalTok{, reg}\FloatTok{.3}\NormalTok{, reg}\FloatTok{.4}\NormalTok{, }\AttributeTok{header =} \ConstantTok{FALSE}\NormalTok{, }\AttributeTok{type =} \StringTok{"text"}\NormalTok{, }\AttributeTok{se =} \FunctionTok{list}\NormalTok{(reg}\FloatTok{.1}\SpecialCharTok{$}\NormalTok{rse, reg}\FloatTok{.2}\SpecialCharTok{$}\NormalTok{rse, reg}\FloatTok{.3}\SpecialCharTok{$}\NormalTok{rse, reg}\FloatTok{.4}\SpecialCharTok{$}\NormalTok{rse))}
\end{Highlighting}
\end{Shaded}

\begin{verbatim}
## 
## ===============================================================================================
##                                                 Dependent variable:                            
##                     ---------------------------------------------------------------------------
##                                                        aauw                                    
##                            (1)                (2)                (3)                (4)        
## -----------------------------------------------------------------------------------------------
## ngirls                   5.058***           5.043***           5.125***           4.987***     
##                          (1.207)            (1.208)            (1.216)            (1.415)      
##                                                                                                
## -----------------------------------------------------------------------------------------------
## Observations              1,735              1,735              1,735              1,735       
## R2                        0.148              0.151              0.155              0.238       
## Adjusted R2               0.138              0.139              0.127              0.110       
## Residual Std. Error 39.354 (df = 1714) 39.330 (df = 1711) 39.605 (df = 1679) 39.982 (df = 1486)
## ===============================================================================================
## Note:                                                               *p<0.1; **p<0.05; ***p<0.01
\end{verbatim}

\textbf{Answer:} the first 2 have have no combined fixed effects. This
means if we wanted to we could see the effects of each variable on the
result. The third has a combined year and region. The downside to the
fourth method is that, since all of the fixed effects are combined, it
may be difficult to tell what the effects of the different variables
are. Additionally, the data may become too specific.

\clearpage

\hypertarget{question-in-her-paper-washington-chooses-not-to-pool-the-data-for-all-four-congresses-and-instead-estimates-her-main-specification-on-each-year-separately.-why-do-you-think-she-makes-this-choice}{%
\subsection{Question: In her paper, Washington chooses not to pool the
data for all four congresses and instead estimates her main
specification on each year separately. Why do you think she makes this
choice?}\label{question-in-her-paper-washington-chooses-not-to-pool-the-data-for-all-four-congresses-and-instead-estimates-her-main-specification-on-each-year-separately.-why-do-you-think-she-makes-this-choice}}

\textbf{Answer:} she examines each year seperatly because she does not
want the variance between the different congresses effecting her
results. She can come to a more accurate conclusion looking at each
individual congress.

\clearpage

\hypertarget{question-check-to-see-that-names-uniquely-identify-each-congress-person.-if-you-are-not-sure-if-they-do-make-a-unique-identifier-for-each-congress-person.}{%
\subsection{Question: Check to see that names uniquely identify each
congress person. If you are not sure if they do, make a unique
identifier for each congress
person.}\label{question-check-to-see-that-names-uniquely-identify-each-congress-person.-if-you-are-not-sure-if-they-do-make-a-unique-identifier-for-each-congress-person.}}

\textbf{Answer:} They are all unique for each givin congress. If they
werent there would be a value that was equal TRUE \textbf{Code:}

\begin{Shaded}
\begin{Highlighting}[]
\NormalTok{name\_105 }\OtherTok{\textless{}{-}}\NormalTok{ basic\_105}\SpecialCharTok{$}\NormalTok{name}
\NormalTok{basic\_106 }\OtherTok{\textless{}{-}} \FunctionTok{filter}\NormalTok{(basic, congress }\SpecialCharTok{==} \StringTok{"106"}\NormalTok{)}
\NormalTok{basic\_107 }\OtherTok{\textless{}{-}} \FunctionTok{filter}\NormalTok{(basic, congress }\SpecialCharTok{==} \StringTok{"107"}\NormalTok{)}
\NormalTok{basic\_108 }\OtherTok{\textless{}{-}} \FunctionTok{filter}\NormalTok{(basic, congress }\SpecialCharTok{==} \StringTok{"108"}\NormalTok{)}
\NormalTok{name\_106 }\OtherTok{\textless{}{-}}\NormalTok{ basic\_106}\SpecialCharTok{$}\NormalTok{name}
\NormalTok{name\_107 }\OtherTok{\textless{}{-}}\NormalTok{ basic\_107}\SpecialCharTok{$}\NormalTok{name}
\NormalTok{name\_108 }\OtherTok{\textless{}{-}}\NormalTok{ basic\_108}\SpecialCharTok{$}\NormalTok{name}
\FunctionTok{duplicated}\NormalTok{(name\_105)}
\end{Highlighting}
\end{Shaded}

\begin{verbatim}
##   [1] FALSE FALSE FALSE FALSE FALSE FALSE FALSE FALSE FALSE FALSE FALSE FALSE
##  [13] FALSE FALSE FALSE FALSE FALSE FALSE FALSE FALSE FALSE FALSE FALSE FALSE
##  [25] FALSE FALSE FALSE FALSE FALSE FALSE FALSE FALSE FALSE FALSE FALSE FALSE
##  [37] FALSE FALSE FALSE FALSE FALSE FALSE FALSE FALSE FALSE FALSE FALSE FALSE
##  [49] FALSE FALSE FALSE FALSE FALSE FALSE FALSE FALSE FALSE FALSE FALSE FALSE
##  [61] FALSE FALSE FALSE FALSE FALSE FALSE FALSE FALSE FALSE FALSE FALSE FALSE
##  [73] FALSE FALSE FALSE FALSE FALSE FALSE FALSE FALSE FALSE FALSE FALSE FALSE
##  [85] FALSE FALSE FALSE FALSE FALSE FALSE FALSE FALSE FALSE FALSE FALSE FALSE
##  [97] FALSE FALSE FALSE FALSE FALSE FALSE FALSE FALSE FALSE FALSE FALSE FALSE
## [109] FALSE FALSE FALSE FALSE FALSE FALSE FALSE FALSE FALSE FALSE FALSE FALSE
## [121] FALSE FALSE FALSE FALSE FALSE FALSE FALSE FALSE FALSE FALSE FALSE FALSE
## [133] FALSE FALSE FALSE FALSE FALSE FALSE FALSE FALSE FALSE FALSE FALSE FALSE
## [145] FALSE FALSE FALSE FALSE FALSE FALSE FALSE FALSE FALSE FALSE FALSE FALSE
## [157] FALSE FALSE FALSE FALSE FALSE FALSE FALSE FALSE FALSE FALSE FALSE FALSE
## [169] FALSE FALSE FALSE FALSE FALSE FALSE FALSE FALSE FALSE FALSE FALSE FALSE
## [181] FALSE FALSE FALSE FALSE FALSE FALSE FALSE FALSE FALSE FALSE FALSE FALSE
## [193] FALSE FALSE FALSE FALSE FALSE FALSE FALSE FALSE FALSE FALSE FALSE FALSE
## [205] FALSE FALSE FALSE FALSE FALSE FALSE FALSE FALSE FALSE FALSE FALSE FALSE
## [217] FALSE FALSE FALSE FALSE FALSE FALSE FALSE FALSE FALSE FALSE FALSE FALSE
## [229] FALSE FALSE FALSE FALSE FALSE FALSE FALSE FALSE FALSE FALSE FALSE FALSE
## [241] FALSE FALSE FALSE FALSE FALSE FALSE FALSE FALSE FALSE FALSE FALSE FALSE
## [253] FALSE FALSE FALSE FALSE FALSE FALSE FALSE FALSE FALSE FALSE FALSE FALSE
## [265] FALSE FALSE FALSE FALSE FALSE FALSE FALSE FALSE FALSE FALSE FALSE FALSE
## [277] FALSE FALSE FALSE FALSE FALSE FALSE FALSE FALSE FALSE FALSE FALSE FALSE
## [289] FALSE FALSE FALSE FALSE FALSE FALSE FALSE FALSE FALSE FALSE FALSE FALSE
## [301] FALSE FALSE FALSE FALSE FALSE FALSE FALSE FALSE FALSE FALSE FALSE FALSE
## [313] FALSE FALSE FALSE FALSE FALSE FALSE FALSE FALSE FALSE FALSE FALSE FALSE
## [325] FALSE FALSE FALSE FALSE FALSE FALSE FALSE FALSE FALSE FALSE FALSE FALSE
## [337] FALSE FALSE FALSE FALSE FALSE FALSE FALSE FALSE FALSE FALSE FALSE FALSE
## [349] FALSE FALSE FALSE FALSE FALSE FALSE FALSE FALSE FALSE FALSE FALSE FALSE
## [361] FALSE FALSE FALSE FALSE FALSE FALSE FALSE FALSE FALSE FALSE FALSE FALSE
## [373] FALSE FALSE FALSE FALSE FALSE FALSE FALSE FALSE FALSE FALSE FALSE FALSE
## [385] FALSE FALSE FALSE FALSE FALSE FALSE FALSE FALSE FALSE FALSE FALSE FALSE
## [397] FALSE FALSE FALSE FALSE FALSE FALSE FALSE FALSE FALSE FALSE FALSE FALSE
## [409] FALSE FALSE FALSE FALSE FALSE FALSE FALSE FALSE FALSE FALSE FALSE FALSE
## [421] FALSE FALSE FALSE FALSE FALSE FALSE FALSE FALSE FALSE FALSE FALSE FALSE
## [433] FALSE FALSE FALSE
\end{verbatim}

\begin{Shaded}
\begin{Highlighting}[]
\FunctionTok{duplicated}\NormalTok{(name\_106)}
\end{Highlighting}
\end{Shaded}

\begin{verbatim}
##   [1] FALSE FALSE FALSE FALSE FALSE FALSE FALSE FALSE FALSE FALSE FALSE FALSE
##  [13] FALSE FALSE FALSE FALSE FALSE FALSE FALSE FALSE FALSE FALSE FALSE FALSE
##  [25] FALSE FALSE FALSE FALSE FALSE FALSE FALSE FALSE FALSE FALSE FALSE FALSE
##  [37] FALSE FALSE FALSE FALSE FALSE FALSE FALSE FALSE FALSE FALSE FALSE FALSE
##  [49] FALSE FALSE FALSE FALSE FALSE FALSE FALSE FALSE FALSE FALSE FALSE FALSE
##  [61] FALSE FALSE FALSE FALSE FALSE FALSE FALSE FALSE FALSE FALSE FALSE FALSE
##  [73] FALSE FALSE FALSE FALSE FALSE FALSE FALSE FALSE FALSE FALSE FALSE FALSE
##  [85] FALSE FALSE FALSE FALSE FALSE FALSE FALSE FALSE FALSE FALSE FALSE FALSE
##  [97] FALSE FALSE FALSE FALSE FALSE FALSE FALSE FALSE FALSE FALSE FALSE FALSE
## [109] FALSE FALSE FALSE FALSE FALSE FALSE FALSE FALSE FALSE FALSE FALSE FALSE
## [121] FALSE FALSE FALSE FALSE FALSE FALSE FALSE FALSE FALSE FALSE FALSE FALSE
## [133] FALSE FALSE FALSE FALSE FALSE FALSE FALSE FALSE FALSE FALSE FALSE FALSE
## [145] FALSE FALSE FALSE FALSE FALSE FALSE FALSE FALSE FALSE FALSE FALSE FALSE
## [157] FALSE FALSE FALSE FALSE FALSE FALSE FALSE FALSE FALSE FALSE FALSE FALSE
## [169] FALSE FALSE FALSE FALSE FALSE FALSE FALSE FALSE FALSE FALSE FALSE FALSE
## [181] FALSE FALSE FALSE FALSE FALSE FALSE FALSE FALSE FALSE FALSE FALSE FALSE
## [193] FALSE FALSE FALSE FALSE FALSE FALSE FALSE FALSE FALSE FALSE FALSE FALSE
## [205] FALSE FALSE FALSE FALSE FALSE FALSE FALSE FALSE FALSE FALSE FALSE FALSE
## [217] FALSE FALSE FALSE FALSE FALSE FALSE FALSE FALSE FALSE FALSE FALSE FALSE
## [229] FALSE FALSE FALSE FALSE FALSE FALSE FALSE FALSE FALSE FALSE FALSE FALSE
## [241] FALSE FALSE FALSE FALSE FALSE FALSE FALSE FALSE FALSE FALSE FALSE FALSE
## [253] FALSE FALSE FALSE FALSE FALSE FALSE FALSE FALSE FALSE FALSE FALSE FALSE
## [265] FALSE FALSE FALSE FALSE FALSE FALSE FALSE FALSE FALSE FALSE FALSE FALSE
## [277] FALSE FALSE FALSE FALSE FALSE FALSE FALSE FALSE FALSE FALSE FALSE FALSE
## [289] FALSE FALSE FALSE FALSE FALSE FALSE FALSE FALSE FALSE FALSE FALSE FALSE
## [301] FALSE FALSE FALSE FALSE FALSE FALSE FALSE FALSE FALSE FALSE FALSE FALSE
## [313] FALSE FALSE FALSE FALSE FALSE FALSE FALSE FALSE FALSE FALSE FALSE FALSE
## [325] FALSE FALSE FALSE FALSE FALSE FALSE FALSE FALSE FALSE FALSE FALSE FALSE
## [337] FALSE FALSE FALSE FALSE FALSE FALSE FALSE FALSE FALSE FALSE FALSE FALSE
## [349] FALSE FALSE FALSE FALSE FALSE FALSE FALSE FALSE FALSE FALSE FALSE FALSE
## [361] FALSE FALSE FALSE FALSE FALSE FALSE FALSE FALSE FALSE FALSE FALSE FALSE
## [373] FALSE FALSE FALSE FALSE FALSE FALSE FALSE FALSE FALSE FALSE FALSE FALSE
## [385] FALSE FALSE FALSE FALSE FALSE FALSE FALSE FALSE FALSE FALSE FALSE FALSE
## [397] FALSE FALSE FALSE FALSE FALSE FALSE FALSE FALSE FALSE FALSE FALSE FALSE
## [409] FALSE FALSE FALSE FALSE FALSE FALSE FALSE FALSE FALSE FALSE FALSE FALSE
## [421] FALSE FALSE FALSE FALSE FALSE FALSE FALSE FALSE FALSE FALSE FALSE FALSE
## [433] FALSE FALSE FALSE
\end{verbatim}

\begin{Shaded}
\begin{Highlighting}[]
\FunctionTok{duplicated}\NormalTok{(name\_107)}
\end{Highlighting}
\end{Shaded}

\begin{verbatim}
##   [1] FALSE FALSE FALSE FALSE FALSE FALSE FALSE FALSE FALSE FALSE FALSE FALSE
##  [13] FALSE FALSE FALSE FALSE FALSE FALSE FALSE FALSE FALSE FALSE FALSE FALSE
##  [25] FALSE FALSE FALSE FALSE FALSE FALSE FALSE FALSE FALSE FALSE FALSE FALSE
##  [37] FALSE FALSE FALSE FALSE FALSE FALSE FALSE FALSE FALSE FALSE FALSE FALSE
##  [49] FALSE FALSE FALSE FALSE FALSE FALSE FALSE FALSE FALSE FALSE FALSE FALSE
##  [61] FALSE FALSE FALSE FALSE FALSE FALSE FALSE FALSE FALSE FALSE FALSE FALSE
##  [73] FALSE FALSE FALSE FALSE FALSE FALSE FALSE FALSE FALSE FALSE FALSE FALSE
##  [85] FALSE FALSE FALSE FALSE FALSE FALSE FALSE FALSE FALSE FALSE FALSE FALSE
##  [97] FALSE FALSE FALSE FALSE FALSE FALSE FALSE FALSE FALSE FALSE FALSE FALSE
## [109] FALSE FALSE FALSE FALSE FALSE FALSE FALSE FALSE FALSE FALSE FALSE FALSE
## [121] FALSE FALSE FALSE FALSE FALSE FALSE FALSE FALSE FALSE FALSE FALSE FALSE
## [133] FALSE FALSE FALSE FALSE FALSE FALSE FALSE FALSE FALSE FALSE FALSE FALSE
## [145] FALSE FALSE FALSE FALSE FALSE FALSE FALSE FALSE FALSE FALSE FALSE FALSE
## [157] FALSE FALSE FALSE FALSE FALSE FALSE FALSE FALSE FALSE FALSE FALSE FALSE
## [169] FALSE FALSE FALSE FALSE FALSE FALSE FALSE FALSE FALSE FALSE FALSE FALSE
## [181] FALSE FALSE FALSE FALSE FALSE FALSE FALSE FALSE FALSE FALSE FALSE FALSE
## [193] FALSE FALSE FALSE FALSE FALSE FALSE FALSE FALSE FALSE FALSE FALSE FALSE
## [205] FALSE FALSE FALSE FALSE FALSE FALSE FALSE FALSE FALSE FALSE FALSE FALSE
## [217] FALSE FALSE FALSE FALSE FALSE FALSE FALSE FALSE FALSE FALSE FALSE FALSE
## [229] FALSE FALSE FALSE FALSE FALSE FALSE FALSE FALSE FALSE FALSE FALSE FALSE
## [241] FALSE FALSE FALSE FALSE FALSE FALSE FALSE FALSE FALSE FALSE FALSE FALSE
## [253] FALSE FALSE FALSE FALSE FALSE FALSE FALSE FALSE FALSE FALSE FALSE FALSE
## [265] FALSE FALSE FALSE FALSE FALSE FALSE FALSE FALSE FALSE FALSE FALSE FALSE
## [277] FALSE FALSE FALSE FALSE FALSE FALSE FALSE FALSE FALSE FALSE FALSE FALSE
## [289] FALSE FALSE FALSE FALSE FALSE FALSE FALSE FALSE FALSE FALSE FALSE FALSE
## [301] FALSE FALSE FALSE FALSE FALSE FALSE FALSE FALSE FALSE FALSE FALSE FALSE
## [313] FALSE FALSE FALSE FALSE FALSE FALSE FALSE FALSE FALSE FALSE FALSE FALSE
## [325] FALSE FALSE FALSE FALSE FALSE FALSE FALSE FALSE FALSE FALSE FALSE FALSE
## [337] FALSE FALSE FALSE FALSE FALSE FALSE FALSE FALSE FALSE FALSE FALSE FALSE
## [349] FALSE FALSE FALSE FALSE FALSE FALSE FALSE FALSE FALSE FALSE FALSE FALSE
## [361] FALSE FALSE FALSE FALSE FALSE FALSE FALSE FALSE FALSE FALSE FALSE FALSE
## [373] FALSE FALSE FALSE FALSE FALSE FALSE FALSE FALSE FALSE FALSE FALSE FALSE
## [385] FALSE FALSE FALSE FALSE FALSE FALSE FALSE FALSE FALSE FALSE FALSE FALSE
## [397] FALSE FALSE FALSE FALSE FALSE FALSE FALSE FALSE FALSE FALSE FALSE FALSE
## [409] FALSE FALSE FALSE FALSE FALSE FALSE FALSE FALSE FALSE FALSE FALSE FALSE
## [421] FALSE FALSE FALSE FALSE FALSE FALSE FALSE FALSE FALSE FALSE FALSE FALSE
## [433] FALSE FALSE FALSE
\end{verbatim}

\begin{Shaded}
\begin{Highlighting}[]
\FunctionTok{duplicated}\NormalTok{(name\_108)}
\end{Highlighting}
\end{Shaded}

\begin{verbatim}
##   [1] FALSE FALSE FALSE FALSE FALSE FALSE FALSE FALSE FALSE FALSE FALSE FALSE
##  [13] FALSE FALSE FALSE FALSE FALSE FALSE FALSE FALSE FALSE FALSE FALSE FALSE
##  [25] FALSE FALSE FALSE FALSE FALSE FALSE FALSE FALSE FALSE FALSE FALSE FALSE
##  [37] FALSE FALSE FALSE FALSE FALSE FALSE FALSE FALSE FALSE FALSE FALSE FALSE
##  [49] FALSE FALSE FALSE FALSE FALSE FALSE FALSE FALSE FALSE FALSE FALSE FALSE
##  [61] FALSE FALSE FALSE FALSE FALSE FALSE FALSE FALSE FALSE FALSE FALSE FALSE
##  [73] FALSE FALSE FALSE FALSE FALSE FALSE FALSE FALSE FALSE FALSE FALSE FALSE
##  [85] FALSE FALSE FALSE FALSE FALSE FALSE FALSE FALSE FALSE FALSE FALSE FALSE
##  [97] FALSE FALSE FALSE FALSE FALSE FALSE FALSE FALSE FALSE FALSE FALSE FALSE
## [109] FALSE FALSE FALSE FALSE FALSE FALSE FALSE FALSE FALSE FALSE FALSE FALSE
## [121] FALSE FALSE FALSE FALSE FALSE FALSE FALSE FALSE FALSE FALSE FALSE FALSE
## [133] FALSE FALSE FALSE FALSE FALSE FALSE FALSE FALSE FALSE FALSE FALSE FALSE
## [145] FALSE FALSE FALSE FALSE FALSE FALSE FALSE FALSE FALSE FALSE FALSE FALSE
## [157] FALSE FALSE FALSE FALSE FALSE FALSE FALSE FALSE FALSE FALSE FALSE FALSE
## [169] FALSE FALSE FALSE FALSE FALSE FALSE FALSE FALSE FALSE FALSE FALSE FALSE
## [181] FALSE FALSE FALSE FALSE FALSE FALSE FALSE FALSE FALSE FALSE FALSE FALSE
## [193] FALSE FALSE FALSE FALSE FALSE FALSE FALSE FALSE FALSE FALSE FALSE FALSE
## [205] FALSE FALSE FALSE FALSE FALSE FALSE FALSE FALSE FALSE FALSE FALSE FALSE
## [217] FALSE FALSE FALSE FALSE FALSE FALSE FALSE FALSE FALSE FALSE FALSE FALSE
## [229] FALSE FALSE FALSE FALSE FALSE FALSE FALSE FALSE FALSE FALSE FALSE FALSE
## [241] FALSE FALSE FALSE FALSE FALSE FALSE FALSE FALSE FALSE FALSE FALSE FALSE
## [253] FALSE FALSE FALSE FALSE FALSE FALSE FALSE FALSE FALSE FALSE FALSE FALSE
## [265] FALSE FALSE FALSE FALSE FALSE FALSE FALSE FALSE FALSE FALSE FALSE FALSE
## [277] FALSE FALSE FALSE FALSE FALSE FALSE FALSE FALSE FALSE FALSE FALSE FALSE
## [289] FALSE FALSE FALSE FALSE FALSE FALSE FALSE FALSE FALSE FALSE FALSE FALSE
## [301] FALSE FALSE FALSE FALSE FALSE FALSE FALSE FALSE FALSE FALSE FALSE FALSE
## [313] FALSE FALSE FALSE FALSE FALSE FALSE FALSE FALSE FALSE FALSE FALSE FALSE
## [325] FALSE FALSE FALSE FALSE FALSE FALSE FALSE FALSE FALSE FALSE FALSE FALSE
## [337] FALSE FALSE FALSE FALSE FALSE FALSE FALSE FALSE FALSE FALSE FALSE FALSE
## [349] FALSE FALSE FALSE FALSE FALSE FALSE FALSE FALSE FALSE FALSE FALSE FALSE
## [361] FALSE FALSE FALSE FALSE FALSE FALSE FALSE FALSE FALSE FALSE FALSE FALSE
## [373] FALSE FALSE FALSE FALSE FALSE FALSE FALSE FALSE FALSE FALSE FALSE FALSE
## [385] FALSE FALSE FALSE FALSE FALSE FALSE FALSE FALSE FALSE FALSE FALSE FALSE
## [397] FALSE FALSE FALSE FALSE FALSE FALSE FALSE FALSE FALSE FALSE FALSE FALSE
## [409] FALSE FALSE FALSE FALSE FALSE FALSE FALSE FALSE FALSE FALSE FALSE FALSE
## [421] FALSE FALSE FALSE FALSE FALSE FALSE FALSE FALSE FALSE FALSE FALSE FALSE
## [433] FALSE FALSE FALSE
\end{verbatim}

\clearpage

\hypertarget{question2-pages-because-we-have-data-for-four-congress-sessions-we-may-be-able-to-see-how-an-individual-congress-persons-voting-patterns-change-as-the-number-of-daughters-they-have-changes.-propose-an-estimating-equation-that-would-allow-you-to-estimate-this-run-your-estimation-and-present-your-results.-be-sure-to-define-all-new-variables.-what-do-your-results-tell-you-why}{%
\subsection{Question:(2 pages) Because we have data for four congress
sessions, we may be able to see how an individual congress person's
voting patterns change as the number of daughters they have changes.
Propose an estimating equation that would allow you to estimate this,
run your estimation and present your results. Be sure to define all new
variables. What do your results tell you?
Why?}\label{question2-pages-because-we-have-data-for-four-congress-sessions-we-may-be-able-to-see-how-an-individual-congress-persons-voting-patterns-change-as-the-number-of-daughters-they-have-changes.-propose-an-estimating-equation-that-would-allow-you-to-estimate-this-run-your-estimation-and-present-your-results.-be-sure-to-define-all-new-variables.-what-do-your-results-tell-you-why}}

\textbf{Equation:} We have to build a dataset of congress members whos
number of children changed between different congresses then run a
regression with the number of girls, total children, female, and
republican \[
\begin{aligned}
  aauw_i&=\beta_0+\beta_1ngirls_i+\beta_2totchi+\beta_3female_i+\beta_4repub_i+\epsilon_i\\
\end{aligned}
\]

\textbf{Code:}

\begin{Shaded}
\begin{Highlighting}[]
\CommentTok{\# have to be in congress twice }
\NormalTok{dataCD }\OtherTok{\textless{}{-}}\NormalTok{ basic}
\CommentTok{\# Remove duplicates by single column}
\NormalTok{data\_repeat }\OtherTok{\textless{}{-}}\NormalTok{ dataCD }\SpecialCharTok{\%\textgreater{}\%}
  \FunctionTok{group\_by}\NormalTok{(name) }\SpecialCharTok{\%\textgreater{}\%}
  \FunctionTok{filter}\NormalTok{(}\FunctionTok{n}\NormalTok{()}\SpecialCharTok{\textgreater{}}\DecValTok{1}\NormalTok{) }


\CommentTok{\# Number of daughters changed. Looks at people who have had a change in the number of daughters and have served multiple terms }
\NormalTok{changed\_daughters }\OtherTok{\textless{}{-}}\NormalTok{ data\_repeat }\SpecialCharTok{\%\textgreater{}\%}
  \FunctionTok{group\_by}\NormalTok{(name) }\SpecialCharTok{\%\textgreater{}\%}
  \FunctionTok{filter}\NormalTok{(}\FunctionTok{length}\NormalTok{(}\FunctionTok{unique}\NormalTok{(ngirls)) }\SpecialCharTok{\textgreater{}} \DecValTok{1}\NormalTok{) }\SpecialCharTok{\%\textgreater{}\%}
  \FunctionTok{ungroup}\NormalTok{()}

\CommentTok{\# same as before but looks only after someone has had an additional daughter. Gets rid of whatever they were befores having another daughter while in congress}
\NormalTok{changed\_daughters2 }\OtherTok{\textless{}{-}}\NormalTok{ data\_repeat }\SpecialCharTok{\%\textgreater{}\%}
  \FunctionTok{group\_by}\NormalTok{(name) }\SpecialCharTok{\%\textgreater{}\%}
  \FunctionTok{arrange}\NormalTok{(congress) }\SpecialCharTok{\%\textgreater{}\%}  \CommentTok{\# Sort by year to ensure chronological order}
  \FunctionTok{filter}\NormalTok{(ngirls }\SpecialCharTok{\textgreater{}} \FunctionTok{lag}\NormalTok{(ngirls, }\AttributeTok{default =} \FunctionTok{first}\NormalTok{(ngirls))) }\SpecialCharTok{\%\textgreater{}\%}
  \FunctionTok{ungroup}\NormalTok{()}


\NormalTok{reg}\FloatTok{.1} \OtherTok{\textless{}{-}} \FunctionTok{felm}\NormalTok{(aauw }\SpecialCharTok{\textasciitilde{}}\NormalTok{ ngirls}\SpecialCharTok{+}\NormalTok{totchi}\SpecialCharTok{+}\NormalTok{female}\SpecialCharTok{+}\NormalTok{repub, changed\_daughters)}
\NormalTok{reg}\FloatTok{.2} \OtherTok{\textless{}{-}} \FunctionTok{felm}\NormalTok{(aauw }\SpecialCharTok{\textasciitilde{}}\NormalTok{ ngirls}\SpecialCharTok{+}\NormalTok{totchi}\SpecialCharTok{+}\NormalTok{female}\SpecialCharTok{+}\NormalTok{repub, changed\_daughters2)}
\FunctionTok{print}\NormalTok{(}\FunctionTok{summary}\NormalTok{(reg}\FloatTok{.1}\NormalTok{),}\AttributeTok{digits=}\DecValTok{3}\NormalTok{)}
\end{Highlighting}
\end{Shaded}

\begin{verbatim}
## 
## Call:
##    felm(formula = aauw ~ ngirls + totchi + female + repub, data = changed_daughters) 
## 
## Residuals:
##    Min     1Q Median     3Q    Max 
## -27.77 -12.03  -3.62   7.41  88.31 
## 
## Coefficients:
##             Estimate Std. Error t value Pr(>|t|)    
## (Intercept)    92.05       4.47   20.58   <2e-16 ***
## ngirls          1.94       4.36    0.45     0.66    
## totchi         -1.41       3.37   -0.42     0.68    
## female          9.69       9.16    1.06     0.29    
## repub         -80.02       5.31  -15.07   <2e-16 ***
## ---
## Signif. codes:  0 '***' 0.001 '**' 0.01 '*' 0.05 '.' 0.1 ' ' 1
## 
## Residual standard error: 21.8 on 71 degrees of freedom
## Multiple R-squared(full model): 0.775   Adjusted R-squared: 0.762 
## Multiple R-squared(proj model): 0.775   Adjusted R-squared: 0.762 
## F-statistic(full model):61.2 on 4 and 71 DF, p-value: <2e-16 
## F-statistic(proj model): 61.2 on 4 and 71 DF, p-value: <2e-16
\end{verbatim}

\begin{Shaded}
\begin{Highlighting}[]
\FunctionTok{print}\NormalTok{(}\FunctionTok{summary}\NormalTok{(reg}\FloatTok{.2}\NormalTok{),}\AttributeTok{digits=}\DecValTok{3}\NormalTok{)}
\end{Highlighting}
\end{Shaded}

\begin{verbatim}
## 
## Call:
##    felm(formula = aauw ~ ngirls + totchi + female + repub, data = changed_daughters2) 
## 
## Residuals:
##    Min     1Q Median     3Q    Max 
## -21.44 -12.60  -7.36   9.69  87.50 
## 
## Coefficients:
##             Estimate Std. Error t value Pr(>|t|)    
## (Intercept)   87.689     10.632    8.25  1.1e-07 ***
## ngirls         0.917      9.117    0.10     0.92    
## totchi        -1.021      6.676   -0.15     0.88    
## female        15.249     21.238    0.72     0.48    
## repub        -73.961     10.557   -7.01  1.1e-06 ***
## ---
## Signif. codes:  0 '***' 0.001 '**' 0.01 '*' 0.05 '.' 0.1 ' ' 1
## 
## Residual standard error: 23.9 on 19 degrees of freedom
## Multiple R-squared(full model): 0.731   Adjusted R-squared: 0.675 
## Multiple R-squared(proj model): 0.731   Adjusted R-squared: 0.675 
## F-statistic(full model):12.9 on 4 and 19 DF, p-value: 3e-05 
## F-statistic(proj model): 12.9 on 4 and 19 DF, p-value: 3e-05
\end{verbatim}

\textbf{Answer:} The results are pretty inconclusive from this test. The
only significant coefficient is republican which makes sense. This has
daughters positive but as mentioned earlier it is not significant.

\clearpage

\hypertarget{question-can-you-think-of-any-identification-concerns-with-this-approach}{%
\subsection{Question: Can you think of any identification concerns with
this
approach?}\label{question-can-you-think-of-any-identification-concerns-with-this-approach}}

It is likely that the pool of people is simply too small which would
make it difficult to have a significant coefficient regarding the number
of daughters. Maybe if we used more congresses, a better conclusion
could be made

\textbf{Answer:} \clearpage

\hypertarget{question-2-pages-using-data-from-all-four-congresses-estimate-the-same-specification-as-that-used-in-column-2-of-table-2-with-the-addition-of-year-and-individual-fixed-effects-and-report-your-results.-why-arent-you-able-to-estimate-a-coefficient-for-certain-covariates}{%
\subsection{Question: (2 pages) Using data from all four congresses,
estimate the same specification as that used in column 2 of table 2 with
the addition of year and individual fixed effects and report your
results. Why aren't you able to estimate a coefficient for certain
covariates?}\label{question-2-pages-using-data-from-all-four-congresses-estimate-the-same-specification-as-that-used-in-column-2-of-table-2-with-the-addition-of-year-and-individual-fixed-effects-and-report-your-results.-why-arent-you-able-to-estimate-a-coefficient-for-certain-covariates}}

\textbf{Code:}

\begin{Shaded}
\begin{Highlighting}[]
\CommentTok{\# add age square and srvlng to data}
\NormalTok{basicdata }\OtherTok{\textless{}{-}}\NormalTok{ basic}

\NormalTok{basicdata}\SpecialCharTok{$}\NormalTok{age2 }\OtherTok{\textless{}{-}}\NormalTok{ basicdata}\SpecialCharTok{$}\NormalTok{age}\SpecialCharTok{\^{}}\DecValTok{2}
\NormalTok{basicdata}\SpecialCharTok{$}\NormalTok{srvlng2 }\OtherTok{\textless{}{-}}\NormalTok{ basicdata}\SpecialCharTok{$}\NormalTok{srvlng}\SpecialCharTok{\^{}}\DecValTok{2}
\NormalTok{basicdata}\SpecialCharTok{$}\NormalTok{repub\_i }\OtherTok{\textless{}{-}} \FunctionTok{ifelse}\NormalTok{(basicdata}\SpecialCharTok{$}\NormalTok{party }\SpecialCharTok{==} \DecValTok{2}\NormalTok{, }\DecValTok{1}\NormalTok{, }\DecValTok{0}\NormalTok{)}

\CommentTok{\#reg.2 \textless{}{-} felm(aauw \textasciitilde{} ngirls+as.factor(totchi), finaldata)}

\NormalTok{reg}\FloatTok{.89} \OtherTok{\textless{}{-}} \FunctionTok{felm}\NormalTok{(}\AttributeTok{data =}\NormalTok{ basicdata, aauw }\SpecialCharTok{\textasciitilde{}}\NormalTok{ ngirls}\SpecialCharTok{+}\NormalTok{female}\SpecialCharTok{+}\NormalTok{white}\SpecialCharTok{+}\NormalTok{repub\_i}\SpecialCharTok{+}\NormalTok{age}\SpecialCharTok{+}\NormalTok{age2}\SpecialCharTok{+}\NormalTok{srvlng}\SpecialCharTok{+}\NormalTok{srvlng2}\SpecialCharTok{+}\FunctionTok{as.factor}\NormalTok{(rgroup)}\SpecialCharTok{+}\NormalTok{demvote}\SpecialCharTok{|}\NormalTok{totchi}\SpecialCharTok{+}\NormalTok{region}\SpecialCharTok{+}\NormalTok{congress}\SpecialCharTok{+}\NormalTok{name)}
\end{Highlighting}
\end{Shaded}

\begin{verbatim}
## Warning in chol.default(mat, pivot = TRUE, tol = tol): the matrix is either
## rank-deficient or not positive definite
\end{verbatim}

\begin{Shaded}
\begin{Highlighting}[]
\FunctionTok{stargazer}\NormalTok{(reg}\FloatTok{.89}\NormalTok{, }\AttributeTok{header =} \ConstantTok{FALSE}\NormalTok{, }\AttributeTok{type =} \StringTok{"text"}\NormalTok{, }\AttributeTok{se =} \FunctionTok{list}\NormalTok{(reg}\FloatTok{.89}\SpecialCharTok{$}\NormalTok{rse))}
\end{Highlighting}
\end{Shaded}

\begin{verbatim}
## 
## ===============================================
##                         Dependent variable:    
##                     ---------------------------
##                                aauw            
## -----------------------------------------------
## ngirls                         2.010           
##                               (2.700)          
##                                                
## female                                         
##                               (0.000)          
##                                                
## white                                          
##                               (0.000)          
##                                                
## repub_i                       -3.034           
##                               (4.593)          
##                                                
## age                          10.393***         
##                               (3.947)          
##                                                
## age2                          -0.003           
##                               (0.008)          
##                                                
## srvlng                        -0.990*          
##                               (0.551)          
##                                                
## srvlng2                       0.0004           
##                               (0.011)          
##                                                
## as.factor(rgroup)1                             
##                               (0.000)          
##                                                
## as.factor(rgroup)2                             
##                               (0.000)          
##                                                
## as.factor(rgroup)3                             
##                               (0.000)          
##                                                
## as.factor(rgroup)4                             
##                               (0.000)          
##                                                
## demvote                        0.454           
##                               (8.239)          
##                                                
## -----------------------------------------------
## Observations                   1,735           
## R2                             0.973           
## Adjusted R2                    0.958           
## Residual Std. Error      8.716 (df = 1121)     
## ===============================================
## Note:               *p<0.1; **p<0.05; ***p<0.01
\end{verbatim}

\textbf{Answer:} since we added all of the years and the added fixed
variables we now have to deal with more multicolinearity. \clearpage

\hypertarget{question-which-fixed-effects-from-the-original-specification-are-now-redundant}{%
\subsection{Question: Which fixed effects from the original
specification are now
redundant?}\label{question-which-fixed-effects-from-the-original-specification-are-now-redundant}}

\textbf{Answer:} female, white, and religion are all redundunt because
they are included in the fixed effect \clearpage

\hypertarget{question-can-you-estimate-a-coefficient-for-repub-what-does-this-imply}{%
\subsection{\texorpdfstring{Question: Can you estimate a coefficient for
\(Repub\)? What does this
imply?}{Question: Can you estimate a coefficient for Repub? What does this imply?}}\label{question-can-you-estimate-a-coefficient-for-repub-what-does-this-imply}}

\textbf{Answer:} we can estimate a coefficient for repub which implies
that it is not included in the fixed effect.

\hypertarget{submission-instructions}{%
\section{Submission instructions:}\label{submission-instructions}}

\begin{itemize}
\item
  Since this is a group assignment only one member of the group will
  upload it to gradescope.
\item
  Make sure the final version of your assignment is knit in pdf format
  and uploaded to gradescope. Make sure you have one question response
  per page (unless otherwise indicated) so that question positions align
  with the template in gradescope.The final PDF should be 29 pages long.
\end{itemize}

\end{document}
