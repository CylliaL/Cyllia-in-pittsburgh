% Options for packages loaded elsewhere
\PassOptionsToPackage{unicode}{hyperref}
\PassOptionsToPackage{hyphens}{url}
%
\documentclass[
]{article}
\usepackage{amsmath,amssymb}
\usepackage{iftex}
\ifPDFTeX
  \usepackage[T1]{fontenc}
  \usepackage[utf8]{inputenc}
  \usepackage{textcomp} % provide euro and other symbols
\else % if luatex or xetex
  \usepackage{unicode-math} % this also loads fontspec
  \defaultfontfeatures{Scale=MatchLowercase}
  \defaultfontfeatures[\rmfamily]{Ligatures=TeX,Scale=1}
\fi
\usepackage{lmodern}
\ifPDFTeX\else
  % xetex/luatex font selection
\fi
% Use upquote if available, for straight quotes in verbatim environments
\IfFileExists{upquote.sty}{\usepackage{upquote}}{}
\IfFileExists{microtype.sty}{% use microtype if available
  \usepackage[]{microtype}
  \UseMicrotypeSet[protrusion]{basicmath} % disable protrusion for tt fonts
}{}
\makeatletter
\@ifundefined{KOMAClassName}{% if non-KOMA class
  \IfFileExists{parskip.sty}{%
    \usepackage{parskip}
  }{% else
    \setlength{\parindent}{0pt}
    \setlength{\parskip}{6pt plus 2pt minus 1pt}}
}{% if KOMA class
  \KOMAoptions{parskip=half}}
\makeatother
\usepackage{xcolor}
\usepackage[margin=1in]{geometry}
\usepackage{color}
\usepackage{fancyvrb}
\newcommand{\VerbBar}{|}
\newcommand{\VERB}{\Verb[commandchars=\\\{\}]}
\DefineVerbatimEnvironment{Highlighting}{Verbatim}{commandchars=\\\{\}}
% Add ',fontsize=\small' for more characters per line
\usepackage{framed}
\definecolor{shadecolor}{RGB}{248,248,248}
\newenvironment{Shaded}{\begin{snugshade}}{\end{snugshade}}
\newcommand{\AlertTok}[1]{\textcolor[rgb]{0.94,0.16,0.16}{#1}}
\newcommand{\AnnotationTok}[1]{\textcolor[rgb]{0.56,0.35,0.01}{\textbf{\textit{#1}}}}
\newcommand{\AttributeTok}[1]{\textcolor[rgb]{0.13,0.29,0.53}{#1}}
\newcommand{\BaseNTok}[1]{\textcolor[rgb]{0.00,0.00,0.81}{#1}}
\newcommand{\BuiltInTok}[1]{#1}
\newcommand{\CharTok}[1]{\textcolor[rgb]{0.31,0.60,0.02}{#1}}
\newcommand{\CommentTok}[1]{\textcolor[rgb]{0.56,0.35,0.01}{\textit{#1}}}
\newcommand{\CommentVarTok}[1]{\textcolor[rgb]{0.56,0.35,0.01}{\textbf{\textit{#1}}}}
\newcommand{\ConstantTok}[1]{\textcolor[rgb]{0.56,0.35,0.01}{#1}}
\newcommand{\ControlFlowTok}[1]{\textcolor[rgb]{0.13,0.29,0.53}{\textbf{#1}}}
\newcommand{\DataTypeTok}[1]{\textcolor[rgb]{0.13,0.29,0.53}{#1}}
\newcommand{\DecValTok}[1]{\textcolor[rgb]{0.00,0.00,0.81}{#1}}
\newcommand{\DocumentationTok}[1]{\textcolor[rgb]{0.56,0.35,0.01}{\textbf{\textit{#1}}}}
\newcommand{\ErrorTok}[1]{\textcolor[rgb]{0.64,0.00,0.00}{\textbf{#1}}}
\newcommand{\ExtensionTok}[1]{#1}
\newcommand{\FloatTok}[1]{\textcolor[rgb]{0.00,0.00,0.81}{#1}}
\newcommand{\FunctionTok}[1]{\textcolor[rgb]{0.13,0.29,0.53}{\textbf{#1}}}
\newcommand{\ImportTok}[1]{#1}
\newcommand{\InformationTok}[1]{\textcolor[rgb]{0.56,0.35,0.01}{\textbf{\textit{#1}}}}
\newcommand{\KeywordTok}[1]{\textcolor[rgb]{0.13,0.29,0.53}{\textbf{#1}}}
\newcommand{\NormalTok}[1]{#1}
\newcommand{\OperatorTok}[1]{\textcolor[rgb]{0.81,0.36,0.00}{\textbf{#1}}}
\newcommand{\OtherTok}[1]{\textcolor[rgb]{0.56,0.35,0.01}{#1}}
\newcommand{\PreprocessorTok}[1]{\textcolor[rgb]{0.56,0.35,0.01}{\textit{#1}}}
\newcommand{\RegionMarkerTok}[1]{#1}
\newcommand{\SpecialCharTok}[1]{\textcolor[rgb]{0.81,0.36,0.00}{\textbf{#1}}}
\newcommand{\SpecialStringTok}[1]{\textcolor[rgb]{0.31,0.60,0.02}{#1}}
\newcommand{\StringTok}[1]{\textcolor[rgb]{0.31,0.60,0.02}{#1}}
\newcommand{\VariableTok}[1]{\textcolor[rgb]{0.00,0.00,0.00}{#1}}
\newcommand{\VerbatimStringTok}[1]{\textcolor[rgb]{0.31,0.60,0.02}{#1}}
\newcommand{\WarningTok}[1]{\textcolor[rgb]{0.56,0.35,0.01}{\textbf{\textit{#1}}}}
\usepackage{graphicx}
\makeatletter
\def\maxwidth{\ifdim\Gin@nat@width>\linewidth\linewidth\else\Gin@nat@width\fi}
\def\maxheight{\ifdim\Gin@nat@height>\textheight\textheight\else\Gin@nat@height\fi}
\makeatother
% Scale images if necessary, so that they will not overflow the page
% margins by default, and it is still possible to overwrite the defaults
% using explicit options in \includegraphics[width, height, ...]{}
\setkeys{Gin}{width=\maxwidth,height=\maxheight,keepaspectratio}
% Set default figure placement to htbp
\makeatletter
\def\fps@figure{htbp}
\makeatother
\setlength{\emergencystretch}{3em} % prevent overfull lines
\providecommand{\tightlist}{%
  \setlength{\itemsep}{0pt}\setlength{\parskip}{0pt}}
\setcounter{secnumdepth}{-\maxdimen} % remove section numbering
\ifLuaTeX
  \usepackage{selnolig}  % disable illegal ligatures
\fi
\IfFileExists{bookmark.sty}{\usepackage{bookmark}}{\usepackage{hyperref}}
\IfFileExists{xurl.sty}{\usepackage{xurl}}{} % add URL line breaks if available
\urlstyle{same}
\hypersetup{
  pdftitle={Untitled},
  pdfauthor={Cyllia},
  hidelinks,
  pdfcreator={LaTeX via pandoc}}

\title{Untitled}
\author{Cyllia}
\date{2024-01-23}

\begin{document}
\maketitle

\begin{Shaded}
\begin{Highlighting}[]
\FunctionTok{library}\NormalTok{(}\StringTok{"quantmod"}\NormalTok{) }\CommentTok{\# add quantmod to the list of Packages}
\end{Highlighting}
\end{Shaded}

\begin{verbatim}
## Warning: 程辑包'quantmod'是用R版本4.3.2 来建造的
\end{verbatim}

\begin{verbatim}
## 载入需要的程辑包:xts
\end{verbatim}

\begin{verbatim}
## Warning: 程辑包'xts'是用R版本4.3.2 来建造的
\end{verbatim}

\begin{verbatim}
## 载入需要的程辑包:zoo
\end{verbatim}

\begin{verbatim}
## 
## 载入程辑包:'zoo'
\end{verbatim}

\begin{verbatim}
## The following objects are masked from 'package:base':
## 
##     as.Date, as.Date.numeric
\end{verbatim}

\begin{verbatim}
## 载入需要的程辑包:TTR
\end{verbatim}

\begin{verbatim}
## Warning: 程辑包'TTR'是用R版本4.3.2 来建造的
\end{verbatim}

\begin{verbatim}
## Registered S3 method overwritten by 'quantmod':
##   method            from
##   as.zoo.data.frame zoo
\end{verbatim}

\begin{Shaded}
\begin{Highlighting}[]
\FunctionTok{source}\NormalTok{(}\StringTok{"generate\_data\_functions.R"}\NormalTok{) }\CommentTok{\# add functions from the local R file named generate\_data\_functions.R}
\FunctionTok{source}\NormalTok{(}\StringTok{"ols\_function.R"}\NormalTok{) }\CommentTok{\# add functions from the local R file named ols\_function.R}
\FunctionTok{source}\NormalTok{(}\StringTok{"t\_test\_function.R"}\NormalTok{) }\CommentTok{\# add functions from the local R file named t\_test\_function.R}
\end{Highlighting}
\end{Shaded}

\#\#(a) Compile quarterly data for the U.S. real gross private domestic
investment (DI) from 1947Q1 to 2023Q4.

\begin{Shaded}
\begin{Highlighting}[]
\FunctionTok{getSymbols}\NormalTok{(}\AttributeTok{Symbols =}\StringTok{"GPDI"}\NormalTok{,}\AttributeTok{src =} \StringTok{"FRED"}\NormalTok{, }\AttributeTok{from =} \StringTok{\textquotesingle{}1947/01/01\textquotesingle{}}\NormalTok{)}
\end{Highlighting}
\end{Shaded}

\begin{verbatim}
## [1] "GPDI"
\end{verbatim}

\begin{Shaded}
\begin{Highlighting}[]
\NormalTok{qua\_di }\OtherTok{\textless{}{-}} \FunctionTok{as.matrix}\NormalTok{(GPDI[,}\DecValTok{1}\NormalTok{])}
\NormalTok{qua\_di\_date }\OtherTok{=} \FunctionTok{as.Date}\NormalTok{(}\FunctionTok{row.names}\NormalTok{(qua\_di))}
\NormalTok{n\_obs\_qua\_di }\OtherTok{=} \FunctionTok{length}\NormalTok{(qua\_di\_date)}
\end{Highlighting}
\end{Shaded}

\#\#(b) Compute growth in quarterly DI (GDI), provide its summary
statistics and plot the data.

\begin{Shaded}
\begin{Highlighting}[]
\NormalTok{di\_return }\OtherTok{\textless{}{-}} \FunctionTok{diff}\NormalTok{(qua\_di)}\SpecialCharTok{/}\NormalTok{qua\_di[}\DecValTok{1}\SpecialCharTok{:}\NormalTok{n\_obs\_qua\_di}\DecValTok{{-}1}\NormalTok{,}\DecValTok{1}\NormalTok{]}
\NormalTok{di\_return\_date }\OtherTok{=}\NormalTok{ qua\_di\_date[}\DecValTok{2}\SpecialCharTok{:}\NormalTok{n\_obs\_qua\_di]}
\NormalTok{n\_obs\_qua\_return }\OtherTok{=} \FunctionTok{length}\NormalTok{(di\_return\_date)}

\FunctionTok{plot}\NormalTok{(di\_return, }\AttributeTok{type =} \StringTok{"l"}\NormalTok{, }\AttributeTok{col =} \StringTok{"darkblue"}\NormalTok{, }\AttributeTok{main =} \StringTok{"Growth in Quarterly GDI"}\NormalTok{, }\AttributeTok{ylab =} \StringTok{"GDI Growth"}\NormalTok{)}
\end{Highlighting}
\end{Shaded}

\includegraphics{PS1Q3_files/figure-latex/unnamed-chunk-3-1.pdf}

\#\#(c) Compute and plot empirical autocorrelation function. Given the
plot, do you expect any time-series correlation among the observations?
Explain why?

\begin{Shaded}
\begin{Highlighting}[]
\FunctionTok{acf}\NormalTok{(di\_return,}\AttributeTok{lag=}\FunctionTok{round}\NormalTok{(n\_obs\_qua\_return}\SpecialCharTok{\^{}}\NormalTok{(}\DecValTok{1}\SpecialCharTok{/}\DecValTok{3}\NormalTok{)))}
\end{Highlighting}
\end{Shaded}

\includegraphics{PS1Q3_files/figure-latex/unnamed-chunk-4-1.pdf}

\begin{Shaded}
\begin{Highlighting}[]
\NormalTok{ACF}\OtherTok{=}\FunctionTok{acf}\NormalTok{(di\_return,}\AttributeTok{lag=}\FunctionTok{round}\NormalTok{(n\_obs\_qua\_return}\SpecialCharTok{\^{}}\NormalTok{(}\DecValTok{1}\SpecialCharTok{/}\DecValTok{3}\NormalTok{)), }\AttributeTok{plot =} \ConstantTok{FALSE}\NormalTok{)}
\NormalTok{ACF}\SpecialCharTok{$}\NormalTok{acf}
\end{Highlighting}
\end{Shaded}

\begin{verbatim}
## , , 1
## 
##             [,1]
## [1,]  1.00000000
## [2,]  0.18230712
## [3,]  0.11675470
## [4,] -0.04992628
## [5,] -0.22861728
## [6,] -0.16254377
## [7,] -0.17190977
## [8,] -0.05641716
\end{verbatim}

Yes I do expect some time-series correlation among the observations.
Because there are some lags where the bars extend beyond the dotted
lines.

\#\#(d) Set the maximum number of lags to the integer closest to the
number of observations to the power one-third. Perform a test for joint
autocorrelation in GDI and report your result. Does your finding
consistent with that of Part 3c? Explain why?

\begin{Shaded}
\begin{Highlighting}[]
\NormalTok{t\_ratio }\OtherTok{\textless{}{-}}\NormalTok{ ACF}\SpecialCharTok{$}\NormalTok{acf[}\DecValTok{2}\NormalTok{]}\SpecialCharTok{*}\FunctionTok{sqrt}\NormalTok{(n\_obs\_qua\_return) }
\NormalTok{t\_ratio}
\end{Highlighting}
\end{Shaded}

\begin{verbatim}
## [1] 3.189072
\end{verbatim}

\begin{Shaded}
\begin{Highlighting}[]
\FunctionTok{Box.test}\NormalTok{(di\_return, }\AttributeTok{lag =} \FunctionTok{round}\NormalTok{(n\_obs\_qua\_return}\SpecialCharTok{\^{}}\NormalTok{(}\DecValTok{1}\SpecialCharTok{/}\DecValTok{3}\NormalTok{)), }\AttributeTok{type =} \StringTok{"Ljung{-}Box"}\NormalTok{) }
\end{Highlighting}
\end{Shaded}

\begin{verbatim}
## 
##  Box-Ljung test
## 
## data:  di_return
## X-squared = 50.143, df = 7, p-value = 1.354e-08
\end{verbatim}

\begin{Shaded}
\begin{Highlighting}[]
\FunctionTok{Box.test}\NormalTok{(di\_return, }\AttributeTok{lag =} \FunctionTok{round}\NormalTok{(n\_obs\_qua\_return}\SpecialCharTok{\^{}}\NormalTok{(}\DecValTok{1}\SpecialCharTok{/}\DecValTok{3}\NormalTok{)), }\AttributeTok{type =} \StringTok{"Box{-}Pierce"}\NormalTok{)}
\end{Highlighting}
\end{Shaded}

\begin{verbatim}
## 
##  Box-Pierce test
## 
## data:  di_return
## X-squared = 49.199, df = 7, p-value = 2.074e-08
\end{verbatim}

\#\#(e) Consider an AR(1) model and compute the theoretical
autocorrelation function. Compare your findings with that of Part 3c.

\begin{Shaded}
\begin{Highlighting}[]
\NormalTok{lag\_di\_return }\OtherTok{=} \FunctionTok{rbind}\NormalTok{(}\ConstantTok{NA}\NormalTok{, }\FunctionTok{as.matrix}\NormalTok{(di\_return[}\DecValTok{1}\SpecialCharTok{:}\NormalTok{(n\_obs\_qua\_return}\DecValTok{{-}1}\NormalTok{),}\DecValTok{1}\NormalTok{]))}
\NormalTok{intercept }\OtherTok{=} \FunctionTok{matrix}\NormalTok{(}\DecValTok{1}\NormalTok{,n\_obs\_qua\_return)}
\NormalTok{X }\OtherTok{=} \FunctionTok{cbind}\NormalTok{(intercept,lag\_di\_return)}
\NormalTok{y }\OtherTok{=}\NormalTok{ di\_return}
\NormalTok{reg\_result }\OtherTok{=} \FunctionTok{ols}\NormalTok{(X[}\DecValTok{2}\SpecialCharTok{:}\NormalTok{n\_obs\_qua\_return,],}\FunctionTok{as.matrix}\NormalTok{(y[}\DecValTok{2}\SpecialCharTok{:}\NormalTok{n\_obs\_qua\_return,}\DecValTok{1}\NormalTok{]))}
\DecValTok{1} \SpecialCharTok{{-}} \FunctionTok{sum}\NormalTok{(reg\_result}\SpecialCharTok{$}\NormalTok{u\_hat}\SpecialCharTok{\^{}}\DecValTok{2}\NormalTok{)}\SpecialCharTok{/}\FunctionTok{sum}\NormalTok{(y}\SpecialCharTok{\^{}}\DecValTok{2}\NormalTok{)}
\end{Highlighting}
\end{Shaded}

\begin{verbatim}
## [1] 0.1381177
\end{verbatim}

\begin{Shaded}
\begin{Highlighting}[]
\NormalTok{beta\_hat }\OtherTok{=}\NormalTok{ reg\_result}\SpecialCharTok{$}\NormalTok{beta\_hat}
\NormalTok{beta\_hat}
\end{Highlighting}
\end{Shaded}

\begin{verbatim}
##            [,1]
## [1,] 0.01449087
## [2,] 0.18233574
\end{verbatim}

\begin{Shaded}
\begin{Highlighting}[]
\NormalTok{var\_beta\_hat }\OtherTok{=}\NormalTok{ reg\_result}\SpecialCharTok{$}\NormalTok{var\_beta\_hat}
\NormalTok{test\_result }\OtherTok{=} \FunctionTok{t\_test}\NormalTok{(beta\_hat,var\_beta\_hat)}
\NormalTok{test\_result}\SpecialCharTok{$}\NormalTok{t\_stat}
\end{Highlighting}
\end{Shaded}

\begin{verbatim}
##          [,1]
## [1,] 4.769901
## [2,] 3.244819
\end{verbatim}

\begin{Shaded}
\begin{Highlighting}[]
\NormalTok{test\_result}\SpecialCharTok{$}\NormalTok{p\_value}
\end{Highlighting}
\end{Shaded}

\begin{verbatim}
##              [,1]
## [1,] 1.843163e-06
## [2,] 1.175253e-03
\end{verbatim}

\begin{Shaded}
\begin{Highlighting}[]
\NormalTok{ar\_coeff }\OtherTok{\textless{}{-}} \FunctionTok{as.numeric}\NormalTok{(beta\_hat[}\DecValTok{2}\NormalTok{])}
\NormalTok{ma\_coeff }\OtherTok{\textless{}{-}} \DecValTok{0}
\NormalTok{TACF }\OtherTok{\textless{}{-}} \FunctionTok{ARMAacf}\NormalTok{(ar\_coeff, ma\_coeff, }\AttributeTok{lag.max =} \FunctionTok{round}\NormalTok{(n\_obs\_qua\_return}\SpecialCharTok{\^{}}\NormalTok{(}\DecValTok{1}\SpecialCharTok{/}\DecValTok{3}\NormalTok{))) }
\FunctionTok{plot}\NormalTok{(}\FunctionTok{c}\NormalTok{(}\DecValTok{0}\SpecialCharTok{:}\FunctionTok{round}\NormalTok{(n\_obs\_qua\_return}\SpecialCharTok{\^{}}\NormalTok{(}\DecValTok{1}\SpecialCharTok{/}\DecValTok{3}\NormalTok{))),ACF}\SpecialCharTok{$}\NormalTok{acf,}\AttributeTok{type=}\StringTok{\textquotesingle{}l\textquotesingle{}}\NormalTok{,}\AttributeTok{xlab=}\StringTok{\textquotesingle{}Lag\textquotesingle{}}\NormalTok{,}\AttributeTok{ylab=}\StringTok{\textquotesingle{}ACF\textquotesingle{}}\NormalTok{,}\AttributeTok{ylim=}\FunctionTok{c}\NormalTok{(}\SpecialCharTok{{-}}\FloatTok{0.1}\NormalTok{,}\DecValTok{1}\NormalTok{))}
\FunctionTok{lines}\NormalTok{(}\DecValTok{0}\SpecialCharTok{:}\FunctionTok{round}\NormalTok{(n\_obs\_qua\_return}\SpecialCharTok{\^{}}\NormalTok{(}\DecValTok{1}\SpecialCharTok{/}\DecValTok{3}\NormalTok{)),TACF,}\AttributeTok{lty=}\DecValTok{2}\NormalTok{)}
\FunctionTok{grid}\NormalTok{(}\AttributeTok{nx =} \DecValTok{4}\NormalTok{, }\AttributeTok{ny =} \DecValTok{4}\NormalTok{)}
\end{Highlighting}
\end{Shaded}

\includegraphics{PS1Q3_files/figure-latex/unnamed-chunk-7-1.pdf}

\end{document}
